\newcommand\TEAM{IUT slowFourierTransformation}
\newcommand\UNI{Islamic University of Technology}
\newcommand\COLS{3}
\newcommand\ORT{portrait}
\newcommand\FSZ{10}
\documentclass[FSZ,a4paper,onesided]{article}
\usepackage[utf8]{inputenc}
\usepackage{amsmath}
\usepackage{listings}
\usepackage{graphicx}
\usepackage{multicol}
\usepackage[utf8]{inputenc}
\usepackage[english]{babel}
\usepackage[usenames,dvipsnames]{color}
\usepackage{verbatim}
\usepackage{hyperref}
\usepackage{geometry}
\usepackage{fancyhdr}
\usepackage{titlesec}

\geometry{verbose,\ORT,a4paper,tmargin=1.0cm,bmargin=.5cm,lmargin=.5cm,rmargin=.5cm, headsep=.5cm}

\definecolor{dkgreen}{rgb}{0,0.6,0}
\definecolor{gray}{rgb}{0.5,0.5,0.5}
\definecolor{mauve}{rgb}{0.58,0,0.82}

\lstset{frame=tb,
  language=C++,
  aboveskip=1mm,
  belowskip=1mm,
  showstringspaces=false,
  columns=flexible,
  basicstyle={\footnotesize\ttfamily},
  numbers=none,
  numberstyle=\tiny\color{gray},
  keywordstyle=\color{blue},
  commentstyle=\color{dkgreen},
  stringstyle=\color{mauve},
  breaklines=true,
  breakatwhitespace=false,
  tabsize=1
}

\fancyhf{}
\renewcommand{\headrulewidth}{1pt}
\pagestyle{fancy}
\lhead{\large{\textbf{\TEAM}, \textbf{\UNI}}}
\rhead{\thepage}

\titleformat*{\section}{\large\bfseries}
\titleformat*{\subsection}{\normalsize\bfseries}
\titleformat*{\subsubsection}{\normalsize}
\titlespacing*{\section}
{0pt}{0ex}{0ex}
\titlespacing*{\subsection}
{0pt}{0ex}{0ex}
\titlespacing*{\subsubsection}
{0pt}{0ex}{0ex}
\setlength{\columnsep}{0.05in}
\setlength{\columnseprule}{1px}


\begin{document}

\begin{multicols*}{\COLS}
\pagenumbering{gobble}
\tableofcontents
\newpage
\pagenumbering{arabic}
\lstloadlanguages{C++,Java}
\subsection*{Sublime Build}
\begin{lstlisting}[language= Pascal, commentstyle=\color{black}, numberstyle=\tiny\color{black}, keywordstyle=\color{black}, stringstyle=\color{black},
]
{
    "cmd" : ["g++ -std=c++14 -DSONIC $file_name -o $file_base_name && timeout 4s ./$file_base_name<inputf.in>outputf.in"], 
    "selector" : "source.cpp",
    "file_regex": "^(..[^:]*):([0-9]+):?([0-9]+)?:? (.*)$",
    "shell": true,
    "working_dir" : "$file_path"
}
\end{lstlisting}
\subsection*{vimrc}

\begin{lstlisting}[language= Pascal, commentstyle=\color{black}, numberstyle=\tiny\color{black}, keywordstyle=\color{black}, stringstyle=\color{black},
]
set mouse=a
  set termguicolors
  filetype plugin indent on
  syntax on

" Some useful settings
  set smartindent expandtab ignorecase smartcase incsearch relativenumber nowrap autoread splitright splitbelow
  set tabstop=4         "the width of a tab
  set shiftwidth=4      "the width for indent
  colorscheme torte

"auto pair curlybraces
  inoremap {<ENTER> {}<LEFT><CR><ESC><S-o>

" mapping jj to esc
  inoremap jj <ESC>  

  "compile and run using file input put  
  autocmd filetype cpp map <F5> :wa<CR>:!clear && g++ % -D LOCAL -std=c++17 -Wall -Wextra -Wconversion -Wshadow -Wfloat-equal -o  ~/Codes/prog && (timeout 5 ~/Codes/prog < ~/Codes/in) >  ~/Codes/out<CR>
  "copy to input file
  map <F4> :!xclip -o -sel clip > ~/Codes/in <CR><CR>
  map <F6> :vsplit ~/Codes/in<CR>:split ~/Codes/out<CR><C-w>=20<C-w><<C-w><C-h>

 " Leader key
  let mapleader=',,'

 " Copy template
  noremap <Leader>t :!cp ~/Codes/temp.cpp %<CR><CR>
  :autocmd BufNewFile *.cpp 0r ~/Codes/temp.cpp

  "note if vim-features +clipboard is not found, it will not work
  "for fast check :echo has('clipboard) = 0 if clipboard features not present,
  "need vim-gtk / vim-gtk3 package for this
  set clipboard=unnamedplus
\end{lstlisting}

\subsection*{Stress-tester}
\begin{lstlisting}[language= Pascal, commentstyle=\color{black}, numberstyle=\tiny\color{black}, keywordstyle=\color{black}, stringstyle=\color{black},
]
#!/bin/bash
# Call as stresstester GENERATOR SOL1 SOL2 ITERATIONS [--count]
for i in $(seq 1 "$4") ; do
    [[ $* == *--count* ]] && echo "Attempt $i/$4"
    $1 > in.txt
    $2 < in.txt > out1.txt
    $3 < in.txt > out2.txt
    diff -y out1.txt out2.txt > diff.txt
    if [ $? -ne 0 ] ; then
        echo "Differing Testcase Found:"; cat in.txt
        echo -e "\nOutputs:"; cat diff.txt
        break
    fi
done
\end{lstlisting}


\section{All Macros}
\begin{lstlisting}
//#pragma GCC optimize("Ofast")
//#pragma GCC optimization ("O3")
//#pragma comment(linker, "/stack:200000000")
//#pragma GCC optimize("unroll-loops")
//#pragma GCC target("sse,sse2,sse3,ssse3,sse4,popcnt,abm,mmx,avx,tune=native")
#include <ext/pb_ds/assoc_container.hpp>
#include <ext/pb_ds/tree_policy.hpp>
using namespace __gnu_pbds;
    //find_by_order(k) --> returns iterator to the kth largest element counting from 0
    //order_of_key(val) --> returns the number of items in a set that are strictly smaller than our item
template <typename DT> 
using ordered_set = tree <DT, null_type, less<DT>, rb_tree_tag,tree_order_statistics_node_update>;

// debug template
void __print(int x) {cerr << x;}
void __print(long long x) {cerr << x;}
void __print(unsigned long long x) {cerr << x;}
void __print(double x) {cerr << x;}
void __print(long double x) {cerr << x;}
void __print(char x) {cerr << '\'' << x << '\'';}
void __print(const string &x) {cerr << '\"' << x << '\"';}
void __print(bool x) {cerr << (x ? "true" : "false");}

template<typename T, typename V>
void __print(const pair<T, V> &x) {cerr << '{'; __print(x.first); cerr << ", "; __print(x.second); cerr << '}';}
template<typename T>
void __print(const T &x) {int f = 0; cerr << "{"; for (auto &i: x) cerr << (f++ ? ", " : ""), __print(i); cerr << "}";}
void _print() {cerr << "]\n";}
template <typename T, typename... V>
void _print(T t, V... v) {__print(t); if (sizeof...(v)) cerr << ", "; _print(v...);}
#ifdef SONIC
#define debug(x...) cerr << "[" << #x << "] = ["; _print(x)
#else
#define debug(x...)
#endif

#define fastio            ios_base::sync_with_stdio(0);cin.tie(0);
#define Make(x,p)       (x | (1<<p))
#define DeMake(x,p)     (x & ~(1<<p))
#define Check(x,p)      (x & (1<<p))

template<size_t N>
bitset<N> rotl( std::bitset<N> const& bits, unsigned count ) {
    count %= N;  // Limit count to range [0,N)
    return bits << count | bits >> (N - count);
}\end{lstlisting}
\section{DP}
\subsection{Convex Hull Trick}
\begin{lstlisting}
 struct line {
   ll m, c;
   line() {}
   line(ll m, ll c) : m(m), c(c) {}
 };
 struct convex_hull_trick {
   vector<line>lines;
   int ptr = 0;
   convex_hull_trick() {}
   bool bad(line a, line b, line c) {
     return 1.0 * (c.c - a.c) * (a.m - b.m) < 1.0 * (b.c - a.c) * (a.m - c.m);
   }
   void add(line L) {
     int sz = lines.size();
     while (sz >= 2 && bad(lines[sz - 2], lines[sz - 1], L)) {
       lines.pop_back(); sz--;
     }
     lines.pb(L);
   }
   ll get(int idx, int x) {
     return (1ll * lines[idx].m * x + lines[idx].c);
   }
   ll query(int x) {
     if (lines.empty()) return 0;
     if (ptr >= lines.size()) ptr = lines.size() - 1;
     while (ptr < lines.size() - 1 && get(ptr, x) > get(ptr + 1, x)) ptr++;
     return get(ptr, x);
   }
 };
 ll sum[MAX];
 ll dp[MAX];
 int arr[MAX];
 int main() {
   fastio;
   int t;
   cin >> t;
   while (t--) {
     int n, a, b, c;
     cin >> n >> a >> b >> c;
     for (int i = 1; i <= n; i++) cin >> sum[i];
     for (int i = 1; i <= n; i++) dp[i] = 0, sum[i] += sum[i - 1];
     convex_hull_trick cht;
     cht.add( line(0, 0) );
     for (int pos = 1; pos <= n; pos++) {
       dp[pos] = cht.query(sum[pos]) - 1ll * a * sqr(sum[pos]) - c;
       cht.add( line(2ll * a * sum[pos], dp[pos] - a * sqr(sum[pos])) );
     }
     ll ans = (-1ll * dp[n]);
     ans += (1ll * sum[n] * b);
     cout << ans << "\n";
   }
 } 
\end{lstlisting}
\subsection{Divide and Conquer dp}
\begin{lstlisting}
#include<bits/stdc++.h>
using namespace std;
using LL = long long;
const int K = 805, N = 4005;
LL dp[2][N], _cost[N][N];
// 1-indexed for convenience
LL cost(int l, int r) {
    return _cost[r][r] - _cost[l - 1][r] - _cost[r][l - 1] + _cost[l - 1][l - 1] >> 1;
}
void compute(int cnt, int l, int r, int optl, int optr) {
    if(l > r) return;
    int mid = l + r >> 1;
    LL best = INT_MAX;
    int opt = -1;
    for(int i = optl; i <= min(mid, optr); i++) {
        LL cur = dp[cnt ^ 1][i - 1] + cost(i, mid);
        if(cur < best) best = cur, opt = i;
    }
    dp[cnt][mid] = best;
    compute(cnt, l, mid - 1, optl, opt);
    compute(cnt, mid + 1, r, opt, optr);
}
LL dnc_dp(int k, int n) {
    fill(dp[0] + 1, dp[0] + n + 1, INT_MAX);
    for(int cnt = 1; cnt <= k; cnt++) {
        compute(cnt & 1, 1, n, 1, n);
    }
    return dp[k & 1][n];
}
int main() {
    cin.tie(0) -> sync_with_stdio(0);
    int n, k;
    cin >> n >> k;
    for(int i = 1; i <= n; i++) {
        for(int j = 1; j <= n; j++) {
            cin >> _cost[i][j];
            _cost[i][j] += _cost[i - 1][j] + _cost[i][j - 1] - _cost[i - 1][j - 1];
        }
    }
    cout << dnc_dp(k, n) << '\n';
    return 0;
}
\end{lstlisting}
\subsection{Knuth optimization}
\begin{lstlisting}
#include<bits/stdc++.h>
using namespace std;
using LL = long long;
// SPOJ BRKSTRING
const int N = 1005;
LL dp[N][N], a[N];
int opt[N][N];
LL cost(int i, int j) {
    return a[j + 1] - a[i];
}
LL knuth_optimization(int n) {
    for(int i = 0; i < n; i++) {
        dp[i][i] = 0;
        opt[i][i] = i;
    }
    for(int i = n - 2; i >= 0; i--) {
        for(int j = i + 1; j < n; j++) {
            LL mn = LLONG_MAX;
            LL c = cost(i, j);
            for(int k = opt[i][j - 1]; k <= min(j-1, opt[i + 1][j]); k++) {
                if(mn > dp[i][k] + dp[k + 1][j] + c) {
                    mn = dp[i][k] + dp[k + 1][j] + c;
                    opt[i][j] = k;
                }
            }
            dp[i][j] = mn;
        }
    }
    return dp[0][n - 1];
}
int main() {
    cin.tie(0) -> sync_with_stdio(0);
    int m, n;
    while(cin >> m >> n) {
        for(int i = 1; i <= n; i++) {
            cin >> a[i];
        }
        a[0] = 0, a[n + 1] = m;
        cout << knuth_optimization(n + 1) << '\n';
    }
    
    return 0;
}
\end{lstlisting}
\subsection{Li Chao Tree}
\begin{lstlisting}
struct line {
  ll m, c;
  line(ll m = 0, ll c = 0) : m(m), c(c) {}
};
ll calc(line L, ll x) {
  return 1ll * L.m * x + L.c;
}
struct node {
  ll m, c;
  line L;
  node *lft, *rt;
  node(ll m = 0, ll c = 0, node *lft = NULL, node *rt = NULL) : L(line(m, c)), lft(lft), rt(rt) {}
};
struct LiChao {
  node *root;
  LiChao() {
    root = new node();
  }
  void update(node *now, int L, int R, line newline) {
    int mid = L + (R - L) / 2;
    line lo = now->L, hi = newline;
    if (calc(lo, L) > calc(hi, L)) swap(lo, hi);
    if (calc(lo, R) <= calc(hi, R)) {
      now->L = hi;
      return;
    }
    if (calc(lo, mid) < calc(hi, mid)) {
      now->L = hi;
      if (now->rt == NULL) now->rt = new node();
      update(now->rt, mid + 1, R, lo);
    } else {
      now->L = lo;
      if (now->lft == NULL) now->lft = new node();
      update(now->lft, L, mid, hi);
    }
  }
  ll query(node *now, int L, int R, ll x) {
    if (now == NULL) return -inf;
    int mid = L + (R - L) / 2;
    if (x <= mid) return max( calc(now->L, x), query(now->lft, L, mid, x) );
    else return max( calc(now->L, x), query(now->rt, mid + 1, R, x) );
  }
};
\end{lstlisting}
\subsection{Triangulation DP}
\begin{lstlisting}
bool valid[205][205];
ll dp[205][205];
ll solve(int L, int R) {
  if (L + 1 == R) return 1;
  if (dp[L][R] != -1) return dp[L][R];
  ll ret = 0;
  for (int mid = L + 1; mid < R; mid++) {
    if (valid[L][mid] && valid[mid][R]) {
      ///selecting triangle(P[L], P[mid], P[R])
      ll temp = ( solve(L, mid) * solve(mid, R) ) % MOD;
      ret = (ret + temp) % MOD;
    }
  }
  return dp[L][R] = ret;
}
\end{lstlisting}
\section{Data Structure}
\subsection{BIT-2D}
\begin{lstlisting}
#include "bits/stdc++.h"
using namespace std;
 
const int N = 1008;
int bit[N][N], n, m;
int a[N][N], q;
 
void update(int x, int y, int val) {
    for (; x < N; x += -x & x)
        for (int j = y; j < N; j += -j & j) bit[x][j] += val;
}
 
int get(int x, int y) {
    int ans = 0;
    for (; x; x -= x & -x)
        for (int j = y; j; j -= j & -j) ans += bit[x][j];
    return ans;
}
 
int get(int x1, int y1, int x2, int y2) {
    return get(x2, y2) - get(x1 - 1, y2) - get(x2, y1 - 1) + get(x1 - 1, y1 - 1);
}
\end{lstlisting}
\subsection{BIT}
\begin{lstlisting}
#include "bits/stdc++.h"
using namespace std;

struct BIT {
    int n;
    vector<int> bit; 

    BIT(int n) {
        this->n = n;
        bit.resize(n);
    }
    void update(int x, int delta) {
        for (; x <= n; x += x & -x) bit[x] += delta;
    }

    int query(int x) {
        int sum = 0;
        for (; x > 0; x -= x & -x) sum += bit[x];
        return sum;
    }
};

int main() {}
\end{lstlisting}
\subsection{Binary Trie}
\begin{lstlisting}
const int N = 1e7 + 5, b = 30;
int tc = 1;
struct node{
    int vis = 0;
    int to[2] = {0, 0};
    int val[2] = {0, 0};
    void update() {
        to[0] = to[1] = 0;
        val[0] = val[1] = 0;
        vis = tc;
    }
} T[N + 2];
node *root = T;
int ptr = 0;
node* nxt(node* cur, int x) {
    if(cur -> to[x] == 0) cur -> to[x] = ++ptr;
    assert(ptr < N);
    int idx = cur -> to[x];
    if(T[idx]. vis < tc) T[idx].update();
    return T + idx;
}
int query(int j, int aj) {
    int ans = 0, jaj = j ^ aj;
    node *cur = root;
    for(int k = b - 1; ~k; k--) {
        maximize(ans, nxt(cur, (jaj >> k & 1) ^ 1) -> val[1 ^ (aj >> k & 1)]);
        cur = nxt(cur, (jaj >> k & 1));
    }
    return ans;
}
void insert(int j, int aj, int val) {
    int jaj = j ^ aj;
    node *cur = root;
    for(int k = b - 1; ~k; k--) {
        cur = nxt(cur, (jaj >> k & 1));
        maximize(cur -> val[j >> k & 1], val);
    }
}
void clear() {
    tc++;
    ptr = 0;
    root -> update();
}
\end{lstlisting}
\subsection{DSU With Rollbacks}
\begin{lstlisting}
struct Rollback_DSU {
    int n;
    vector <int> par, sz;
    vector <pair <int, int>> op;
    Rollback_DSU(int n) : par(n), sz(n, 1) {
        iota(par.begin(), par.end(), 0);
        op.reserve(n);
    }
    int Anc(int node) {
        for(; node != par[node]; node = par[node]);
        return node;
    }
    void Unite(int x, int y) {
        if(sz[x = Anc(x)] < sz[y = Anc(y)])
            swap(x, y);
        op.emplace_back(x, y);
        par[y] = x;
        sz[x] += sz[y];
    }
    void Undo(size_t t) {
        for(; op.size() > t; op.pop_back()) {
            par[op.back().second] = op.back().second;
            sz[op.back().first] -= sz[op.back().second];
        }
    }
};\end{lstlisting}
\subsection{DSU on Tree}
\begin{lstlisting}
///Query: Number of distinct names among all the k'th son of a node.
const int N = 100005;
string name[N];
vector<int>G[N];
vector<pii>Q[N];
int L[N],ans[N];

void dfs(int v,int d){
    L[v]=d;
    for(int i:G[v]) dfs(i,d+1);
    return;
}

void dsu(int v,map<int,set<string>>&mp){
    for(int i:G[v]){
        map<int,set<string>>s;
        dsu(i,s);
        if(s.size()>mp.size()) swap(mp,s);
        for(auto it:s) mp[it.ff].insert(all(it.ss));
    }
    if(v!=0) mp[L[v]].insert(name[v]); //Here zero is not a actual node
    for(pii p:Q[v]) ans[p.ss] = mp[p.ff].size();
    return;
}

int main(){
    int n;
    cin >> n;
    FOR(i,1,n){
        int u;
        cin >> name[i] >> u;
        G[u].pb(i);
    }
    dfs(0,0);
    int q;
    cin >>q;
    FOR(i,1,q){
        int v,k;
        cin >> v >> k;
        Q[v].pb(pii(k+L[v],i)); //Actual level
    }
    map<int,set<string>>mp;
    dsu(0,mp);
    FOR(i,1,q) cout << ans[i] << '\n';
    return 0;
}
\end{lstlisting}
\subsection{Dominator Tree}
\begin{lstlisting}
struct dominator {
  int n, d_t;
  vector<vector<int>> g, rg, tree, bucket;
  vector<int> sdom, dom, par, dsu, label, val, rev;
  dominator() {}
  dominator(int n) : 
    n(n), d_t(0), g(n + 1), rg(n + 1),
    tree(n + 1), bucket(n + 1), sdom(n + 1),
    dom(n + 1), par(n + 1), dsu(n + 1),
    label(n + 1), val(n + 1), rev(n + 1)
  { for (int i = 1; i <= n; i++) sdom[i] = dom[i] = dsu[i] = label[i] = i; }

  void add_edge(int u, int v) { g[u].pb(v); }
  int dfs(int u) {
    d_t++;
    val[u] = d_t, rev[d_t] = u;
    label[d_t] = sdom[d_t] = dom[d_t] = d_t;
    for (int v : g[u]) {
      if (!val[v]) {
        dfs(v);
        par[val[v]] = val[u];
      }
      rg[val[v]].pb(val[u]);
    }
  }
  int findpar(int u, int x = 0) {
    if (dsu[u] == u) return x ? -1 : u;
    int v = findpar(dsu[u], x + 1);
    if (v < 0) return u;
    if (sdom[label[dsu[u]]] < sdom[label[u]]) label[u] = label[dsu[u]];
    dsu[u] = v;
    return x ? v : label[u];
  }
  void join(int u, int v) { dsu[v] = u; }
  vector<vector<int>> build(int s) {
    dfs(s);
    for (int i = n; i >= 1; i--) {
      for (int j = 0; j < rg[i].size(); j++) {
        sdom[i] = min(sdom[i], sdom[ findpar(rg[i][j]) ]);
      }
      if (i > 1) bucket[sdom[i]].pb(i);
      for (int w : bucket[i]) {
        int v = findpar(w);
        if (sdom[v] == sdom[w]) dom[w] = sdom[w];
        else dom[w] = v;
      }
      if (i > 1) join(par[i], i);
    }
    for (int i = 2; i <= n; i++) {
      if (dom[i] != sdom[i]) dom[i] = dom[dom[i]];
      tree[rev[i]].pb(rev[dom[i]]);
      tree[rev[dom[i]]].pb(rev[i]);
    }
    return tree;
  }
};
\end{lstlisting}
\subsection{Implicit Segment Tree}
\begin{lstlisting}
struct node {
  int val;
  node *lft, *rt;
  node() {}
  node(int val = 0) : val(val), lft(NULL), rt(NULL) {}
};

struct implicit_segtree {
  node *root;
  implicit_segtree() {}
  implicit_segtree(int n) {
    root = new node(n);
  }
  void update(node *now, int L, int R, int idx, int val) {
    if (L == R) {
      now -> val += val;
      return;
    }
    int mid = L + (R - L) / 2;
    if (now->lft == NULL) now->lft = new node(mid - L + 1);
    if (now->rt == NULL) now->rt = new node(R - mid);
    if (idx <= mid) update(now->lft, L, mid, idx, val);
    else update(now->rt, mid + 1, R, idx, val);
    now->val = (now->lft)->val + (now->rt)->val;
  }

  int query(node *now, int L, int R, int k) {
    if (L == R) return L;
    int mid = L + (R - L) / 2;
    if (now->lft == NULL) now->lft = new node(mid - L + 1);
    if (now->rt == NULL) now->rt = new node(R - mid);
    if (k <= (now->lft)->val) return query(now->lft, L, mid, k);
    else return query(now->rt, mid + 1, R, k - (now->lft)->val);
  }
};
\end{lstlisting}
\subsection{Implicit Treap}
\begin{lstlisting}
mt19937 rnd(chrono::steady_clock::now().time_since_epoch().count());
typedef struct node* pnode;
struct node {
  int prior, sz;
  ll val, sum, lazy;
  bool rev;
  node *lft, *rt;
  node(int val = 0, node *lft = NULL, node *rt = NULL) : lft(lft), rt(rt), prior(rnd()), sz(1), val(val), rev(false), sum(0), lazy(0) {}
};
struct implicit_treap {
  pnode root;
  implicit_treap() {
    root = NULL;
  }
  int get_sz(pnode now) {
    return now ? now->sz : 0;
  }
  void update_sz(pnode now) {
    if (!now) return;
    now->sz = 1 + get_sz(now->lft) + get_sz(now->rt);
  }
  // lazy sum
  void push(pnode now) {
    if (!now || !now->lazy) return;
    now->val += now->lazy;
    now->sum += get_sz(now) * now->lazy;
    if (now->lft) now->lft->lazy += now->lazy;
    if (now->rt) now->rt->lazy += now->lazy;
    now->lazy = 0;
  }
  void combine(pnode now) {
    if (!now) return;
    now->sum = now->val; // reset the node
    push(now->lft), push(now->rt); // update lft and rt
    now->sum += (now->lft ? now->lft->sum : 0) + (now->rt ? now->rt->sum : 0);
  }
  // reverse substring
  void push(pnode now) {
    if (!now || !now->rev) return;
    now->rev = false;
    swap(now->lft, now->rt);
    if (now->lft) now->lft->rev ^= true;
    if (now->rt) now->rt->rev ^= true;
  }
  sort ascending or descending
  void push(pnode now) {
    if (!now || !now->sort_kor) return;
    if (now->sort_kor == -1) swap(now->lft, now->rt);
    int cnt[26];
    for (int i = 0; i < 26; i++) cnt[i] = now->cnt[i];
    int idx = 0;
    if (now->lft) {
      memset(now->lft->cnt, 0, sizeof now->lft->cnt);
      int lft_sz = get_sz(now->lft);
      while (idx < 26 && lft_sz) {
        int mn = min(cnt[idx], lft_sz);
        now->lft->cnt[idx] = mn;
        cnt[idx] -= mn; lft_sz -= mn;
        if (!cnt[idx]) idx++;
      }
      now->lft->sort_kor = now->sort_kor;
    }
    while (!cnt[idx]) idx++;
    now->val = idx, cnt[idx]--;
    if (!cnt[idx]) idx++;
    if (now->rt) {
      memset(now->rt->cnt, 0, sizeof now->rt->cnt);
      int rt_sz = get_sz(now->rt);
      while (idx < 26 && rt_sz) {
        int mn = min(cnt[idx], rt_sz);
        now->rt->cnt[idx] = mn;
        cnt[idx] -= mn; rt_sz -= mn;
        if (!cnt[idx]) idx++;
      }
      now->rt->sort_kor = now->sort_kor;
    }
    if (now->sort_kor == -1) swap(now->lft, now->rt);
    now->sort_kor = 0;
  }
  void combine(pnode now) {
    if (!now) return;
    memset(now->cnt, 0, sizeof now->cnt);
    for (int i = 0; i < 26; i++) {
      now->cnt[i] = (now->lft ? now->lft->cnt[i] : 0) + (now->rt ? now->rt->cnt[i] : 0);
    }
    now->cnt[now->val]++;
  }
  ///first pos ta elements go to left, others go to right
  void split(pnode now, pnode &lft, pnode &rt, int pos, int add = 0) {
    if (!now) return void(lft = rt = NULL);
    push(now);
    int cur = add + get_sz(now->lft);
    if (cur < pos) split(now->rt, now->rt, rt, pos, cur + 1), lft = now;
    else split(now->lft, lft, now->lft, pos, add), rt = now;
    update_sz(now); combine(now);
  }
  void merge(pnode &now, pnode lft, pnode rt) {
    push(lft);
    push(rt);
    if (!lft || !rt) now = lft ? lft : rt;
    else if (lft->prior > rt->prior) merge(lft->rt, lft->rt, rt), now = lft;
    else merge(rt->lft, lft, rt->lft), now = rt;
    update_sz(now); combine(now);
  }
  void insert(int pos, ll val) {
    if (!root) return void(root = new node(val));
    pnode lft, rt;
    split(root, lft, rt, pos - 1);
    pnode notun = new node(val);
    merge(root, lft, notun);
    merge(root, root, rt);
  }
  void erase(int pos) {
    pnode lft, rt, temp;
    split(root, lft, rt, pos);
    split(lft, lft, temp, pos - 1);
    merge(root, lft, rt);
    delete(temp);
  }
  void reverse(int l, int r) {
    pnode lft, rt, mid;
    split(root, lft, mid, l - 1);
    split(mid, mid, rt, r - l + 1);
    mid->rev ^= true;
    merge(root, lft, mid);
    merge(root, root, rt);
  }
  void right_shift(int l, int r) {
    pnode lft, rt, mid, last;
    split(root, lft, mid, l - 1);
    split(mid, mid, rt, r - l + 1);
    split(mid, mid, last, r - l);
    merge(mid, last, mid);
    merge(root, lft, mid);
    merge(root, root, rt);
  }
  void output(pnode now, vector<int>&v) {
    if (!now) return;
    push(now);
    output(now->lft, v);
    v.pb(now->val);
    output(now->rt, v);
  }
  vector<int>get_arr() {
    vector<int>ret;
    output(root, ret);
    return ret;
  }
};
\end{lstlisting}
\subsection{Link Cut Tree}
\begin{lstlisting}
struct SplayTree {
  struct node {
    int ch[2] = {0, 0}, p = 0;
    ll self = 0, path = 0;
    ll sub = 0, extra = 0;
    bool rev = false;
  };
  vector<node> T;
  SplayTree(int n) : T(n + 1) {}
  void push(int x) {
    if (!x) return;
    int l = T[x].ch[0], r = T[x].ch[1];
    if (T[x].rev) {
      T[l].rev ^= true, T[r].rev ^= true;
      swap(T[x].ch[0], T[x].ch[1]);
      T[x].rev = false;
    }
  }
  void pull(int x) {
    int l = T[x].ch[0], r = T[x].ch[1];
    push(l), push(r);
    T[x].path = T[x].self + T[l].path + T[r].path;
    T[x].sub = T[x].self + T[x].extra + T[l].sub + T[r].sub;
  }
  void set(int parent, int child, int d) {
    T[parent].ch[d] = child;
    T[child].p = parent;
    pull(parent);
  }
  int dir(int x) {
    int parent = T[x].p;
    if (!parent) return -1;
    return (T[parent].ch[0] == x) ? 0 : (T[parent].ch[1] == x) ? 1 : -1;
  }
  void rotate(int x) {
    int parent = T[x].p, gparent = T[parent].p;
    int dx = dir(x), dp = dir(parent);
    set(parent, T[x].ch[!dx], dx);
    set(x, parent, !dx);
    if (~dp) set(gparent, x, dp);
    T[x].p = gparent;
  }
  void splay(int x) {
    push(x);
    while (~dir(x)) {
      int parent = T[x].p;
      int gparent = T[parent].p;
      push(gparent), push(parent), push(x);
      int dx = dir(x), dp = dir(parent);
      if (~dp) rotate(dx != dp ? x : parent);
      rotate(x);
    }
  }
};
struct LinkCut : SplayTree {
  LinkCut(int n) : SplayTree(n) {}
  void cut_right(int x) {
    splay(x);
    int r = T[x].ch[1];
    T[x].extra += T[r].sub;
    T[x].ch[1] = 0, pull(x);
  }
  int access(int x) {
    int u = x, v = 0;
    for (; u; v = u, u = T[u].p) {
      cut_right(u);
      T[u].extra -= T[v].sub;
      T[u].ch[1] = v, pull(u);
    }
    return splay(x), v;
  }
  void make_root(int x) {
    access(x);
    T[x].rev ^= true, push(x);
  }
  void link(int u, int v) {
    make_root(v), access(u);
    T[u].extra += T[v].sub;
    T[v].p = u, pull(u);
  }
  void cut(int u) {
    access(u);
    T[u].ch[0] = T[ T[u].ch[0] ].p = 0;
    pull(u);
  }
  void cut(int u, int v) {
    make_root(u), access(v);
    T[v].ch[0] = T[u].p = 0, pull(v);
  }
  int find_root(int u) {
    access(u), push(u);
    while (T[u].ch[0]) {
      u = T[u].ch[0], push(u);
    }
    return splay(u), u;
  }
  int lca(int u, int v) {
    if (u == v) return u;
    access(u);
    int ret = access(v);
    return T[u].p ? ret : 0;
  }
  // subtree query of u if v is the root
  ll subtree(int u, int v) {
    make_root(v), access(u);
    return T[u].self + T[u].extra;
  }
  ll path(int u, int v) {
    make_root(u), access(v);
    return T[v].path;
  }
  // point update
  void update(int u, ll val) {
    access(u);
    T[u].self = val, pull(u);
  }
};
\end{lstlisting}
\subsection{MO with Update}
\begin{lstlisting}
const int N = 1e5 + 5, sz = 2700, bs = 25;
int arr[N], freq[2 * N], cnt[2*N], id[N], ans[N];
struct query{
    int l, r, t, L, R;
    query(int l = 1, int r = 0, int t = 1, int id = -1) : l(l), r(r), t(t), L(l / sz), R(r / sz) {}
    bool operator < (const query &rhs) const {
        return (L < rhs.L) or (L == rhs.L and R < rhs.R) or (L ==  rhs.L and R == rhs.R and t < rhs.t);
    }
} Q[N];
struct update{
    int idx, val, last;
} Up[N];
int qi = 0, ui = 0;
int l = 1, r = 0, t = 0;

void add(int idx) {
    --cnt[freq[arr[idx]]];
    freq[arr[idx]]++;
    cnt[freq[arr[idx]]]++;
}
void remove(int idx){
    --cnt[freq[arr[idx]]];
    freq[arr[idx]]--;
    cnt[freq[arr[idx]]]++;
}
void apply(int t) {
    const bool f = l <= Up[t].idx and Up[t].idx <= r;
    if(f) remove(Up[t].idx);
    arr[Up[t].idx] = Up[t].val;
    if(f) add(Up[t].idx);
}
void undo(int t) {
    const bool f = l <= Up[t].idx and Up[t].idx <= r;
    if(f) remove(Up[t].idx);
    arr[Up[t].idx] = Up[t].last;
    if(f) add(Up[t].idx);
}
int mex(){
    for(int i = 1; i <= N; i++)
        if(!cnt[i])
            return i;
    assert(0);
}
int main() {
    int n, q;
    cin >> n >> q;
    int counter = 0;
    map <int, int> M;
    for(int i = 1; i <= n; i++){
        cin >> arr[i];
        if(!M[arr[i]])
            M[arr[i]] = ++counter;
        arr[i] = M[arr[i]];
    }
    iota(id, id + N, 0);
    while(q--){
        int tp, x, y;
        cin >> tp >> x >> y;
        if(tp == 1) Q[++qi] = query(x, y, ui);
        else {
            if(!M[y]) M[y] = ++counter;
            y = M[y];
            Up[++ui] = {x, y, arr[x]};
            arr[x] = y;
        }
    }
    t = ui;
    cnt[0] = 3 * n;
    sort(id + 1, id + qi + 1, [&](int x, int y) {return Q[x] < Q[y];});
    for(int i = 1; i <= qi; i++) {
        int x = id[i];
        while(Q[x].t > t) apply(++t);
        while(Q[x].t < t) undo(t--);
        while(Q[x].l < l) add(--l);
        while(Q[x].r > r) add(++r);
        while(Q[x].l > l) remove(l++);
        while(Q[x].r < r) remove(r--);
        ans[x] = mex();
    }
    for(int i = 1; i <= qi; i++)
        cout << ans[i] << '\n';
}\end{lstlisting}
\subsection{Merge Sort Tree}
\begin{lstlisting}
vector<LL>Tree[4*MAXN];
LL arr[MAXN];

vector<LL> merge(vector<LL> v1, vector<LL> v2)
{
    LL i = 0, j = 0;
    vector<LL> ret;

    while(i < v1.size() || j < v2.size())
    {
        if(i == v1.size())
        {
            ret.push_back(v2[j]);
            j++;
        }
        else if(j == v2.size())
        {
            ret.push_back(v1[i]);
            i++;
        }
        else
        {
            if(v1[i] < v2[j])
            {
                ret.push_back(v1[i]);
                i++;
            }
            else
            {
                ret.push_back(v2[j]);
                j++;
            }
        }
    }

    return ret;
}

void Build(LL node, LL bg, LL ed)
{
    if(bg == ed)
    {
        Tree[node].push_back(arr[bg]);
        return;
    }

    LL leftNode = 2*node, rightNode = 2*node + 1;
    LL mid = (bg+ed)/2;

    Build(leftNode, bg, mid);
    Build(rightNode, mid+1, ed);

    Tree[node] = merge(Tree[leftNode], Tree[rightNode]);
}

LL query(LL node, LL bg, LL ed, LL l, LL r, LL k)
{
    if(ed < l || bg > r)
        return 0;

    if(l <= bg && ed <= r)
        return upper_bound(Tree[node].begin(), Tree[node].end(), k) - Tree[node].begin();

    LL leftNode = 2*node, rightNode = 2*node + 1;
    LL mid = (bg + ed)/2;

    return query(leftNode, bg, mid, l, r, k) + query(rightNode, mid+1, ed, l, r, k);
}
\end{lstlisting}
\subsection{Merge Sort Tree.cpp}
\subsection{Persistent Segment Tree}
\begin{lstlisting}
struct Node 
{
    Node *l, *r;
    int sum;
 
    Node(int val) : l(nullptr), r(nullptr), sum(val) {}
    Node(Node *l, Node *r) : l(l), r(r), sum(0) {
        if (l) sum += l->sum;
        if (r) sum += r->sum;
    }
};
 
int a[MAXN];
Node *root[MAXN];
 
Node* Build(int bg, int ed) 
{
    if (bg == ed)
        return new Node(a[bg]);
    int mid = (bg + ed) / 2;
    return new Node(Build(bg, mid), Build(mid+1, ed));
}
 
int Query(Node* v, int bg, int ed, int l, int r) 
{
    if (l > ed || r < bg)
        return 0;
    if (l <= bg && ed <= r)
        return v->sum;
    int mid = (bg + ed) / 2;
    return Query(v->l, bg, mid, l, r) + Query(v->r, mid+1, ed, l, r);
}
 
Node* Update(Node* v, int bg, int ed, int pos, int new_val) 
{
    if (bg == ed)
        return new Node(v->sum + new_val);
    int mid = (bg + ed) / 2;
    if (pos <= mid)
        return new Node(Update(v->l, bg, mid, pos, new_val), v->r);
    else
        return new Node(v->l, Update(v->r, mid+1, ed, pos, new_val));
}\end{lstlisting}
\subsection{RMQ(1D)}
\begin{lstlisting}
/// Source: Folklore

#include <bits/stdc++.h>
using namespace std;
const int N = 1e5 + 9, K = 18;
int st[K][N], a[N], lg[N];

void buildRMQ(int n) {
    for (int i = 1; i <= n; i++) st[0][i] = a[i];

    for (int k = 1; k < K; k++)
        for (int i = 1; i + (1 << k) - 1 <= n; i++)
            st[k][i] = min(st[k - 1][i], st[k - 1][i + (1 << (k - 1))]);

    for (int i = 2; i <= n; i++) lg[i] = lg[i / 2] + 1;
}

int query(int i, int j) {
    int k = lg[j - i + 1];
    return min(st[k][i], st[k][j - (1 << k) + 1]);
}
\end{lstlisting}
\subsection{Segment Tree}
\begin{lstlisting}
constexpr DT I = 0; 
constexpr LT None = 0;
DT val[4 * N];
LT lazy[4 * N];
int L, R;
void pull(int s, int e, int node) {
    val[node] = val[node << 1] + val[node << 1 | 1];
}
void apply(const LT &U, int s, int e, int node) {
    val[node] += (e - s + 1) * U;
    lazy[node] += U;
}
void reset(int node) {
    lazy[node] = None;
}
DT merge(const DT &a, const DT &b) {
    return a + b;
}
DT get(int s, int e, int node) {
    return val[node];
}
void push(int s, int e, int node) {
    if(s == e) return;
    apply(lazy[node], s, s + e >> 1, node << 1);
    apply(lazy[node], s + e + 2 >> 1, e, node << 1 | 1);
    reset(node);
}
void update(int S,int E, LT uval, int s = L, int e = R, int node = 1) {
    if(S > E) return;
    if(S == s and E == e) {
        apply(uval, s, e, node);
        return;
    }
    push(s, e, node);
    int m = s + e >> 1;
    update(S, min(m, E), uval,  s,  m, node * 2);
    update(max(S, m + 1), E, uval, m + 1, e, node * 2 + 1);
    pull(s, e, node);
}
DT query(int S, int E, int s = L, int e = R, int node = 1) {
    if(S > E) return I;
    if(s == S and e == E) return get(s, e, node);
    push(s, e, node);
    int m = s + e >> 1;
    DT L = query(S, min(m, E), s, m, node * 2);
    DT R = query(max(S, m + 1), E, m + 1, e, node * 2 + 1);
    return merge(L, R);
}
void init(int _L, int _R, vector <DT> &v) {
    L = _L, R = _R;
    build(L, R, v);
}
\end{lstlisting}
\subsection{Segtree}
\begin{lstlisting}
#include <bits/stdc++.h>
#define LL long long
using namespace std;

const int N = 1e5 + 7;
int a[N];
LL tr[4 * N];
LL lz[4 * N];

/// 1. Merge left and right
LL combine(LL left, LL right) { return left + right; }

/// 2. Push lazy down and merge lazy
void propagate(int u, int st, int en) {
    if (!lz[u]) return;
    tr[u] += (en - st + 1) * lz[u];
    if (st != en) {
        lz[2 * u] += lz[u];
        lz[2 * u + 1] += lz[u];
    }
    lz[u] = 0;
}

void build(int u, int st, int en) {
    if (st == en) {
        tr[u] = a[st];  /// 3. Initialize
        lz[u] = 0;
    } else {
        int mid = (st + en) / 2;
        build(2 * u, st, mid);
        build(2 * u + 1, mid + 1, en);
        tr[u] = combine(tr[2 * u], tr[2 * u + 1]);
        lz[u] = 0;  /// 3. Initialize
    }
}

void update(int u, int st, int en, int l, int r, int x) {
    propagate(u, st, en);
    if (r < st || en < l)
        return;
    else if (l <= st && en <= r) {
        lz[u] += x;  /// 4. Merge lazy
        propagate(u, st, en);
    } else {
        int mid = (st + en) / 2;
        update(2 * u, st, mid, l, r, x);
        update(2 * u + 1, mid + 1, en, l, r, x);
        tr[u] = combine(tr[2 * u], tr[2 * u + 1]);
    }
}

LL query(int u, int st, int en, int l, int r) {
    propagate(u, st, en);
    if (r < st || en < l)
        return 0;  /// 5. Proper null value
    else if (l <= st && en <= r)
        return tr[u];
    else {
        int mid = (st + en) / 2;
        return combine(query(2 * u, st, mid, l, r),
                       query(2 * u + 1, mid + 1, en, l, r));
    }
}

void debug(int u, int st, int en) {
    cout << "--->" << u << " " << st << " " << en << " " << tr[u] << " "
         << lz[u] << endl;
    if (st == en) return;
    int mid = (st + en) / 2;
    debug(2 * u, st, mid);
    debug(2 * u + 1, mid + 1, en);
}

\end{lstlisting}
\subsection{Treap}
\begin{lstlisting}
mt19937 rnd(chrono::steady_clock::now().time_since_epoch().count());
typedef struct node* pnode;
struct node {
  int prior, val, sz;
  ll sum;
  node *lft, *rt;
  node(int val = 0, node *lft = NULL, node *rt = NULL) : 
    lft(lft), rt(rt), prior(rnd()), val(val), sz(1), sum(0) {}
};
struct treap {
  pnode root;
  treap() {
    root = NULL;
  }
  int get_sz(pnode now) {
    return now ? now->sz : 0;
  }
  void update_sz(pnode now) {
    if (!now) return;
    now->sz = 1 + get_sz(now->lft) + get_sz(now->rt);
  }
  ll get(pnode now) {
    return now ? now->sum : 0;
  }
  void push(pnode now) {}
  void combine(pnode now) {
    if (!now) return;
    now->sum = now->val + get(now->lft) + get(now->rt);
  }
  pnode unite(pnode lft, pnode rt) {
    if (!lft || !rt) return lft ? lft : rt;
    // push(lft), push(rt); this not tested
    if (lft->prior < rt->prior) swap(lft, rt);
    pnode l, r;
    split(rt, l, r, lft->val);
    lft->lft = unite(lft->lft, l), update_sz(lft);
    lft->rt = unite(lft->rt, r), update_sz(lft);
    // combine(lft); this not tested
    return lft;
  }
  ///value < val goes to left, value >= val goes to right
  void split(pnode now, pnode &lft, pnode &rt, int val, int add = 0) {
    push(now);
    if (!now) return void(lft = rt = NULL);
    if (now->val < val) split(now->rt, now->rt, rt, val), lft = now;
    else split(now->lft, lft, now->lft, val), rt = now;
    update_sz(now), combine(now);
  }
  void merge(pnode &now, pnode lft, pnode rt) {
    push(lft), push(rt);
    if (!lft || !rt) now = lft ? lft : rt;
    else if (lft->prior > rt->prior) merge(lft->rt, lft->rt, rt), now = lft;
    else merge(rt->lft, lft, rt->lft), now = rt;
    update_sz(now), combine(now);
  }
  void insert(pnode &now, pnode notun) {
    if (!now) return void(now = notun);
    push(now);
    if (notun->prior > now->prior) split(now, notun->lft, notun->rt, notun->val), now = notun;
    else insert(notun->val < now->val ? now->lft : now->rt, notun);
    update_sz(now), combine(now);
  }
  void erase(pnode &now, int val) {
    push(now);
    if (now->val == val) {
      pnode temp = now;
      merge(now, now->lft, now->rt);
      delete(temp);
    } else erase(val < now->val ? now->lft : now->rt, val);
    update_sz(now), combine(now);
  }
  int get_idx(pnode &now, int val) {
    if (!now) return INT_MIN;
    else if (now->val == val) return 1 + get_sz(now->lft);
    else if (val < now->val) return get_idx(now->lft, val);
    else return (1 + get_sz(now->lft) + get_idx(now->rt, val));
  }
  int find_kth(pnode &now, int k) {
    if (k < 1 || k > get_sz(now)) return -1;
    if (get_sz(now->lft) + 1 == k) return now->val;
    if (k <= get_sz(now->lft)) return find_kth(now->lft, k);
    return find_kth(now->rt, k - get_sz(now->lft) - 1);
  }
  ll prefix_sum(pnode &now, int k) {
    if (k < 1 || k > get_sz(now)) return -inf;
    if (get_sz(now->lft) + 1 == k) return get(now->lft) + now->val;
    if (k <= get_sz(now->lft)) return prefix_sum(now->lft, k);
    return get(now->lft) + now->val + prefix_sum(now->rt, k - get_sz(now->lft) - 1);
  }
  pnode get_rng(int l, int r) { ///gets all l <= values <= r
    pnode lft, rt, mid;
    split(root, lft, mid, l);
    split(mid, mid, rt, r + 1);
    merge(root, lft, rt);
    return mid;
  }
  void output(pnode now, vector<int>&v) {
    if (!now) return;
    output(now->lft, v);
    v.pb(now->val);
    output(now->rt, v);
  }
  vector<int>get_arr() {
    vector<int>ret;
    output(root, ret);
    return ret;
  }
};
\end{lstlisting}
\section{Geometry}
\subsection{Circular}
\begin{lstlisting}
// Extremely inaccurate for finding near touches
// compute intersection of line l with circle c
// The intersections are given in order of the ray (l.a, l.b)
vector<Point> circleLineIntersection(Circle c, Line l) {
    static_assert(is_same<Tf, Ti>::value);
    vector<Point> ret;
    Point b = l.b - l.a, a = l.a - c.o;

    Tf A = dot(b, b), B = dot(a, b);
    Tf C = dot(a, a) - c.r * c.r, D = B*B - A*C;
    if (D < -EPS) return ret;

    ret.push_back(l.a + b * (-B - sqrt(D + EPS)) / A);
    if (D > EPS)
        ret.push_back(l.a + b * (-B + sqrt(D)) / A);
    return ret;
}

// signed area of intersection of circle(c.o, c.r) &&
// triangle(c.o, s.a, s.b) [cross(a-o, b-o)/2]
Tf circleTriangleIntersectionArea(Circle c, Segment s) {
    using Linear::distancePointSegment;
    Tf OA = length(c.o - s.a);
    Tf OB = length(c.o - s.b);

    // sector
    if(dcmp(distancePointSegment(c.o, s) - c.r) >= 0)
        return angleBetween(s.a-c.o, s.b-c.o) * (c.r * c.r) / 2.0;

    // triangle
    if(dcmp(OA - c.r) <= 0 && dcmp(OB - c.r) <= 0)
        return cross(c.o - s.b, s.a - s.b) / 2.0;

    // three part: (A, a) (a, b) (b, B)
    vector<Point> Sect = circleLineIntersection(c, s);
    return circleTriangleIntersectionArea(c, Segment(s.a, Sect[0]))
        + circleTriangleIntersectionArea(c, Segment(Sect[0], Sect[1]))
        + circleTriangleIntersectionArea(c, Segment(Sect[1], s.b));
}

// area of intersecion of circle(c.o, c.r) && simple polyson(p[])
// Tested : https://codeforces.com/gym/100204/problem/F - Little Mammoth
Tf circlePolyIntersectionArea(Circle c, Polygon p) {
    Tf res = 0;
    int n = p.size();
    for(int i = 0; i < n; ++i)
        res += circleTriangleIntersectionArea(c, Segment(p[i], p[(i + 1) % n]));
    return abs(res);
}

// locates circle c2 relative to c1
// interior             (d < R - r)         ----> -2
// interior tangents (d = R - r)         ----> -1
// concentric        (d = 0)
// secants             (R - r < d < R + r) ---->  0
// exterior tangents (d = R + r)         ---->  1
// exterior             (d > R + r)         ---->  2
int circleCirclePosition(Circle c1, Circle c2) {
    Tf d = length(c1.o - c2.o);
    int in = dcmp(d - abs(c1.r - c2.r)), ex = dcmp(d - (c1.r + c2.r));
    return in < 0 ? -2 : in == 0 ? -1 : ex == 0 ? 1 : ex > 0 ? 2 : 0;
}

// compute the intersection points between two circles c1 && c2
vector<Point> circleCircleIntersection(Circle c1, Circle c2) {
    static_assert(is_same<Tf, Ti>::value);

    vector<Point> ret;
    Tf d = length(c1.o - c2.o);
    if(dcmp(d) == 0) return ret;
    if(dcmp(c1.r + c2.r - d) < 0) return ret;
    if(dcmp(abs(c1.r - c2.r) - d) > 0) return ret;

    Point v = c2.o - c1.o;
    Tf co = (c1.r * c1.r + sqLength(v) - c2.r * c2.r) / (2 * c1.r * length(v));
    Tf si = sqrt(abs(1.0 - co * co));
    Point p1 = scale(rotatePrecise(v, co, -si), c1.r) + c1.o;
    Point p2 = scale(rotatePrecise(v, co, si), c1.r) + c1.o;

    ret.push_back(p1);
    if(p1 != p2) ret.push_back(p2);
    return ret;
}

// intersection area between two circles c1, c2
Tf circleCircleIntersectionArea(Circle c1, Circle c2) {
    Point AB = c2.o - c1.o;
    Tf d = length(AB);
    if(d >= c1.r + c2.r) return 0;
    if(d + c1.r <= c2.r) return PI * c1.r * c1.r;
    if(d + c2.r <= c1.r) return PI * c2.r * c2.r;

    Tf alpha1 = acos((c1.r * c1.r + d * d - c2.r * c2.r) / (2.0 * c1.r * d));
    Tf alpha2 = acos((c2.r * c2.r + d * d - c1.r * c1.r) / (2.0 * c2.r * d));
    return c1.sector(2 * alpha1) + c2.sector(2 * alpha2);
}

// returns tangents from a point p to circle c
vector<Point> pointCircleTangents(Point p, Circle c) {
    static_assert(is_same<Tf, Ti>::value);

    vector<Point> ret;
    Point u = c.o - p;
    Tf d = length(u);
    if(d < c.r) ;
    else if(dcmp(d - c.r) == 0) {
        ret = { rotate(u, PI / 2) };
    }
    else {
        Tf ang = asin(c.r / d);
        ret = { rotate(u, -ang), rotate(u, ang) };
    }
    return ret;
}

// returns the points on tangents that touches the circle
vector<Point> pointCircleTangencyPoints(Point p, Circle c) {
    static_assert(is_same<Tf, Ti>::value);

    Point u = p - c.o;
    Tf d = length(u);
    if(d < c.r) return {};
    else if(dcmp(d - c.r) == 0)     return {c.o + u};
    else {
        Tf ang = acos(c.r / d);
        u = u / length(u) * c.r;
        return { c.o + rotate(u, -ang), c.o + rotate(u, ang) };
    }
}

// for two circles c1 && c2, returns two list of points a && b
// such that a[i] is on c1 && b[i] is c2 && for every i
// Line(a[i], b[i]) is a tangent to both circles
// CAUTION: a[i] = b[i] in case they touch | -1 for c1 = c2
int circleCircleTangencyPoints(Circle c1, Circle c2, vector<Point> &a, vector<Point> &b) {
    a.clear(), b.clear();
    int cnt = 0;
    if(dcmp(c1.r - c2.r) < 0) {
        swap(c1, c2); swap(a, b);
    }
    Tf d2 = sqLength(c1.o - c2.o);
    Tf rdif = c1.r - c2.r, rsum = c1.r + c2.r;
    if(dcmp(d2 - rdif * rdif) < 0) return 0;
    if(dcmp(d2) == 0 && dcmp(c1.r - c2.r) == 0) return -1;

    Tf base = angle(c2.o - c1.o);
    if(dcmp(d2 - rdif * rdif) == 0) {
        a.push_back(c1.point(base));
        b.push_back(c2.point(base));
        cnt++;
        return cnt;
    }

    Tf ang = acos((c1.r - c2.r) / sqrt(d2));
    a.push_back(c1.point(base + ang));
    b.push_back(c2.point(base + ang));
    cnt++;
    a.push_back(c1.point(base - ang));
    b.push_back(c2.point(base - ang));
    cnt++;

    if(dcmp(d2 - rsum * rsum) == 0) {
        a.push_back(c1.point(base));
        b.push_back(c2.point(PI + base));
        cnt++;
    }
    else if(dcmp(d2 - rsum * rsum) > 0) {
        Tf ang = acos((c1.r + c2.r) / sqrt(d2));
        a.push_back(c1.point(base + ang));
        b.push_back(c2.point(PI + base + ang));
        cnt++;
        a.push_back(c1.point(base - ang));
        b.push_back(c2.point(PI + base - ang));
        cnt++;
    }
    return cnt;
}\end{lstlisting}
\subsection{Convex}
\begin{lstlisting}
///minkowski sum of two polygons in O(n)
Polygon minkowskiSum(Polygon A, Polygon B){
    int n = A.size(), m = B.size();
    rotate(A.begin(), min_element(A.begin(), A.end()), A.end());
    rotate(B.begin(), min_element(B.begin(), B.end()), B.end());

    A.push_back(A[0]); B.push_back(B[0]);
    for(int i = 0; i < n; i++) A[i] = A[i+1] - A[i];
    for(int i = 0; i < m; i++) B[i] = B[i+1] - B[i];

    Polygon C(n+m+1);
    C[0] = A.back() + B.back();
    merge(A.begin(), A.end()-1, B.begin(), B.end()-1, C.begin()+1, polarComp(Point(0, 0), Point(0, -1)));
    for(int i = 1; i < C.size(); i++) C[i] = C[i] + C[i-1];
    C.pop_back();
    return C;
}

/// finds the rectangle with minimum area enclosing a convex polygon and
/// the rectangle with minimum perimeter enclosing a convex polygon
/// Tested on https://open.kattis.com/problems/fenceortho
pair< Tf, Tf >rotatingCalipersBoundingBox(const Polygon &p) {
    using Linear::distancePointLine;
    static_assert(is_same<Tf, Ti>::value);
    int n = p.size();
    int l = 1, r = 1, j = 1;
    Tf area = 1e100;
    Tf perimeter = 1e100;
    for(int i = 0; i < n; i++) {
        Point v = (p[(i+1)%n] - p[i]) / length(p[(i+1)%n] - p[i]);
        while(dcmp(dot(v, p[r%n] - p[i]) - dot(v, p[(r+1)%n] - p[i])) < 0) r++;
        while(j < r || dcmp(cross(v, p[j%n] - p[i]) - cross(v, p[(j+1)%n] - p[i])) < 0) j++;
        while(l < j || dcmp(dot(v, p[l%n] - p[i]) - dot(v, p[(l+1)%n] - p[i])) > 0) l++;
        Tf w = dot(v, p[r%n] - p[i]) - dot(v, p[l%n] - p[i]);
        Tf h = distancePointLine(p[j%n], Line(p[i], p[(i+1)%n]));
        area = min(area, w * h);
        perimeter = min(perimeter, 2 * w + 2 * h);
    }
    return make_pair(area, perimeter);
}

// returns the left side of polygon u after cutting it by ray a->b
Polygon cutPolygon(Polygon u, Point a, Point b) {
    using Linear::lineLineIntersection;
    using Linear::onSegment;

    Polygon ret;
    int n = u.size();
    for(int i = 0; i < n; i++) {
        Point c = u[i], d = u[(i + 1) % n];
        if(dcmp(cross(b-a, c-a)) >= 0) ret.push_back(c);
        if(dcmp(cross(b-a, d-c)) != 0) {
            Point t;
            lineLineIntersection(a, b - a, c, d - c, t);
            if(onSegment(t, Segment(c, d))) ret.push_back(t);
        }
    }
    return ret;
}

// returns true if point p is in or on triangle abc
bool pointInTriangle(Point a, Point b, Point c, Point p) {
    return dcmp(cross(b - a, p - a)) >= 0
        && dcmp(cross(c - b, p - b)) >= 0
        && dcmp(cross(a - c, p - c)) >= 0;
}

// Tested : https://www.spoj.com/problems/INOROUT
// pt must be in ccw order with no three collinear points
// returns inside = -1, on = 0, outside = 1
int pointInConvexPolygon(const Polygon &pt, Point p) {
    int n = pt.size();
    assert(n >= 3);

    int lo = 1, hi = n - 1;
    while(hi - lo > 1) {
        int mid = (lo + hi) / 2;
        if(dcmp(cross(pt[mid] - pt[0], p - pt[0])) > 0) lo = mid;
        else    hi = mid;
    }

    bool in = pointInTriangle(pt[0], pt[lo], pt[hi], p);
    if(!in) return 1;

    if(dcmp(cross(pt[lo] - pt[lo - 1], p - pt[lo - 1])) == 0) return 0;
    if(dcmp(cross(pt[hi] - pt[lo], p - pt[lo])) == 0) return 0;
    if(dcmp(cross(pt[hi] - pt[(hi + 1) % n], p - pt[(hi + 1) % n])) == 0) return 0;
    return -1;
}

// Extreme Point for a direction is the farthest point in that direction
// also https://codeforces.com/blog/entry/48868
// u is the direction for extremeness
// weakly tested on https://open.kattis.com/problems/fenceortho
int extremePoint(const Polygon &poly, Point u) {
    int n = (int) poly.size();
    int a = 0, b = n;
    while(b - a > 1) {
        int c = (a + b) / 2;
        if(dcmp(dot(poly[c] - poly[(c + 1) % n], u)) >= 0 && dcmp(dot(poly[c] - poly[(c - 1 + n) % n], u)) >= 0) {
            return c;
        }

        bool a_up = dcmp(dot(poly[(a + 1) % n] - poly[a], u)) >= 0;
        bool c_up = dcmp(dot(poly[(c + 1) % n] - poly[c], u)) >= 0;
        bool a_above_c = dcmp(dot(poly[a] - poly[c], u)) > 0;

        if(a_up && !c_up) b = c;
        else if(!a_up && c_up) a = c;
        else if(a_up && c_up) {
            if(a_above_c) b = c;
            else a = c;
        }
        else {
            if(!a_above_c) b = c;
            else a = c;
        }
    }

    if(dcmp(dot(poly[a] - poly[(a + 1) % n], u)) > 0 && dcmp(dot(poly[a] - poly[(a - 1 + n) % n], u)) > 0)
        return a;
    return b % n;
}

// For a convex polygon p and a line l, returns a list of segments
// of p that touch or intersect line l.
// the i'th segment is considered (p[i], p[(i + 1) modulo |p|])
// #1 If a segment is collinear with the line, only that is returned
// #2 Else if l goes through i'th point, the i'th segment is added
// Complexity: O(lg |p|)
vector<int> lineConvexPolyIntersection(const Polygon &p, Line l) {
    assert((int) p.size() >= 3);
    assert(l.a != l.b);

    int n = p.size();
    vector<int> ret;

    Point v = l.b - l.a;
    int lf = extremePoint(p, rotate90(v));
    int rt = extremePoint(p, rotate90(v) * Ti(-1));
    int olf = orient(l.a, l.b, p[lf]);
    int ort = orient(l.a, l.b, p[rt]);

    if(!olf || !ort) {
        int idx = (!olf ? lf : rt);
        if(orient(l.a, l.b, p[(idx - 1 + n) % n]) == 0)
            ret.push_back((idx - 1 + n) % n);
        else    ret.push_back(idx);
        return ret;
    }
    if(olf == ort) return ret;

    for(int i=0; i<2; ++i) {
        int lo = i ? rt : lf;
        int hi = i ? lf : rt;
        int olo = i ? ort : olf;

        while(true) {
            int gap = (hi - lo + n) % n;
            if(gap < 2) break;

            int mid = (lo + gap / 2) % n;
            int omid = orient(l.a, l.b, p[mid]);
            if(!omid) {
                lo = mid;
                break;
            }
            if(omid == olo) lo = mid;
            else hi = mid;
        }
        ret.push_back(lo);
    }
    return ret;
}

// Tested : https://toph.co/p/cover-the-points
// Calculate [ACW, CW] tangent pair from an external point
constexpr int CW = -1, ACW = 1;
bool isGood(Point u, Point v, Point Q, int dir) { return orient(Q, u, v) != -dir; }
Point better(Point u, Point v, Point Q, int dir) { return orient(Q, u, v) == dir ? u : v; }
Point pointPolyTangent(const Polygon &pt, Point Q, int dir, int lo, int hi) {
    while(hi - lo > 1) {
        int mid = (lo + hi) / 2;
        bool pvs = isGood(pt[mid], pt[mid - 1], Q, dir);
        bool nxt = isGood(pt[mid], pt[mid + 1], Q, dir);

        if(pvs && nxt) return pt[mid];
        if(!(pvs || nxt)) {
            Point p1 = pointPolyTangent(pt, Q, dir, mid + 1, hi);
            Point p2 = pointPolyTangent(pt, Q, dir, lo, mid - 1);
            return better(p1, p2, Q, dir);
        }

        if(!pvs) {
            if(orient(Q, pt[mid], pt[lo]) == dir)               hi = mid - 1;
            else if(better(pt[lo], pt[hi], Q, dir) == pt[lo])   hi = mid - 1;
            else    lo = mid + 1;
        }
        if(!nxt) {
            if(orient(Q, pt[mid], pt[lo]) == dir)               lo = mid + 1;
            else if(better(pt[lo], pt[hi], Q, dir) == pt[lo])   hi = mid - 1;
            else    lo = mid + 1;
        }
    }

    Point ret = pt[lo];
    for(int i = lo + 1; i <= hi; i++) ret = better(ret, pt[i], Q, dir);
    return ret;
}
// [ACW, CW] Tangent
pair<Point, Point> pointPolyTangents(const Polygon &pt, Point Q) {
    int n = pt.size();
    Point acw_tan = pointPolyTangent(pt, Q, ACW, 0, n - 1);
    Point cw_tan = pointPolyTangent(pt, Q, CW, 0, n - 1);
    return make_pair(acw_tan, cw_tan);
}\end{lstlisting}
\subsection{Enclosing Circle}
\begin{lstlisting}
// returns false if points are collinear, true otherwise
// circle p touch each arm of triangle abc
bool inscribedCircle(Point a, Point b, Point c, Circle &p) {
    using Linear::distancePointLine;
    static_assert(is_same<Tf, Ti>::value);
    if(orient(a, b, c) == 0) return false;
    Tf u = length(b - c);
    Tf v = length(c - a);
    Tf w = length(a - b);
    p.o = (a * u + b * v + c * w) / (u + v + w);
    p.r = distancePointLine(p.o, Line(a, b));
    return true;
}

// set of points A(x, y) such that PA : QA = rp : rq
Circle apolloniusCircle(Point P, Point Q, Tf rp, Tf rq) {
    static_assert(is_same<Tf, Ti>::value);
    rq *= rq; rp *= rp;
    Tf a = rq - rp;
    assert(dcmp(a));
    Tf g = (rq * P.x - rp * Q.x)/a;
    Tf h = (rq * P.y - rp * Q.y)/a;
    Tf c = (rq * P.x * P.x - rp * Q.x * Q.x + rq * P.y * P.y - rp * Q.y * Q.y)/a;
    Point o(g, h);
    Tf R = sqrt(g * g + h * h - c);
    return Circle(o, R);
}

// returns false if points are collinear, true otherwise
// circle p goes through points a, b && c
bool circumscribedCircle(Point a, Point b, Point c, Circle &p) {
    using Linear::lineLineIntersection;
    if(orient(a, b, c) == 0) return false;
    Point d = (a + b) / 2, e = (a + c) / 2;
    Point vd = rotate90(b - a), ve = rotate90(a - c);
    bool f = lineLineIntersection(d, vd, e, ve, p.o);
    if(f) p.r = length(a - p.o);
    return f;
}

// Following three methods implement Welzl's algorithm for
// the smallest enclosing circle problem: Given a set of
// points, find out the minimal circle that covers them all.
// boundary(p) determines (if possible) a circle that goes
// through the points in p. Ideally |p| <= 3.
// welzl() is a recursive helper function doing the most part
// of Welzl's algorithm. Call minidisk with the set of points
// Randomized Complexity: O(CN) with C~10 (practically lesser)

Circle boundary(const vector<Point> &p) {
    Circle ret;
    int sz = p.size();
    if(sz == 0)         ret.r = 0;
    else if(sz == 1)    ret.o = p[0], ret.r = 0;
    else if(sz == 2)    ret.o = (p[0] + p[1]) / 2, ret.r = length(p[0] - p[1]) / 2;
    else if(!circumscribedCircle(p[0], p[1], p[2], ret))    ret.r = 0;
    return ret;
}
Circle welzl(const vector<Point> &p, int fr, vector<Point> &b) {
    if(fr >= (int) p.size() || b.size() == 3)   return boundary(b);

    Circle c = welzl(p, fr + 1, b);
    if(!c.contains(p[fr])) {
        b.push_back(p[fr]);
        c = welzl(p, fr + 1, b);
        b.pop_back();
    }
    return c;
}
Circle minidisk(vector<Point> p) {
    random_shuffle(p.begin(), p.end());
    vector<Point> q;
    return welzl(p, 0, q);
}\end{lstlisting}
\subsection{Half Planar}
\begin{lstlisting}
using Linear::lineLineIntersection;
struct DirLine {
    Point p, v;
    Tf ang;
    DirLine() {}
    /// Directed line containing point P in the direction v
    DirLine(Point p, Point v) : p(p), v(v) { ang = atan2(v.y, v.x); }
    /// Directed Line for ax+by+c >=0
    DirLine(Tf a, Tf b, Tf c) {
        assert(dcmp(a) || dcmp(b));
        p = dcmp(a) ? Point(-c/a, 0) : Point(0,-c/b);
        v = Point(b, -a);
        ang = atan2(v.y, v.x);
    }
    bool operator<(const DirLine& u) const { return ang < u.ang; }
    bool onLeft(Point x) const { return dcmp(cross(v, x-p)) >= 0; }
};

// Returns the region bounded by the left side of some directed lines
// MAY CONTAIN DUPLICATE POINTS
// OUTPUT IS UNDEFINED if intersection is unbounded
// Complexity: O(n log n) for sorting, O(n) afterwards
Polygon halfPlaneIntersection(vector<DirLine> li) {
    int n = li.size(), first = 0, last = 0;
    sort(li.begin(), li.end());
    vector<Point> p(n);
    vector<DirLine> q(n);
    q[0] = li[0];

    for(int i = 1; i < n; i++) {
        while(first < last && !li[i].onLeft(p[last - 1])) last--;
        while(first < last && !li[i].onLeft(p[first])) first++;
        q[++last] = li[i];
        if(dcmp(cross(q[last].v, q[last-1].v)) == 0) {
            last--;
            if(q[last].onLeft(li[i].p)) q[last] = li[i];
        }
        if(first < last)
            lineLineIntersection(q[last - 1].p, q[last - 1].v, q[last].p, q[last].v, p[last - 1]);
    }

    while(first < last && !q[first].onLeft(p[last - 1])) last--;
    if(last - first <= 1) return {};
    lineLineIntersection(q[last].p, q[last].v, q[first].p, q[first].v, p[last]);
    return Polygon(p.begin()+first, p.begin()+last+1);
}

// O(n^2 lg n) implementation of Voronoi Diagram bounded by INF square
// returns region, where regions[i] = set of points for which closest
// point is site[i]. This region is a polygon.
const Tf INF = 1e10;
vector<Polygon> voronoi(const vector<Point> &site, Tf bsq) {
    int n = site.size();
    vector<Polygon> region(n);
    Point A(-bsq, -bsq), B(bsq, -bsq), C(bsq, bsq), D(-bsq, bsq);

    for(int i = 0; i < n; ++i) {
        vector<DirLine> li(n - 1);
        for(int j = 0, k = 0; j < n; ++j) {
            if(i == j) continue;
            li[k++] = DirLine((site[i] + site[j]) / 2, rotate90(site[j] - site[i]));
        }
        li.emplace_back(A, B - A);
        li.emplace_back(B, C - B);
        li.emplace_back(C, D - C);
        li.emplace_back(D, A - D);
        region[i] = halfPlaneIntersection(li);
    }
    return region;
}\end{lstlisting}
\subsection{Intersecting Segments}
\begin{lstlisting}
// Given a list of segments v, finds a pair (i, j)
// st v[i], v[j] intersects. If none, returns {-1, -1}
// Tested Timus 1469, CF 1359F
struct Event {
    Tf x;
    int tp, id;
    bool operator < (const Event &p) const {
        if(dcmp(x - p.x)) return x < p.x;
        return tp > p.tp;
    }
};

pair<int, int> anyIntersection(const vector<Segment> &v) {
    using Linear::segmentsIntersect;
    static_assert(is_same<Tf, Ti>::value);

    vector<Event> ev;
    for(int i=0; i<v.size(); i++) {
        ev.push_back({min(v[i].a.x, v[i].b.x), +1, i});
        ev.push_back({max(v[i].a.x, v[i].b.x), -1, i});
    }
    sort(ev.begin(), ev.end());

    auto comp = [&v] (int i, int j) {
        Segment p = v[i], q = v[j];
        Tf x = max(min(p.a.x, p.b.x), min(q.a.x, q.b.x));
        auto yvalSegment = [&x](const Line &s) {
            if(dcmp(s.a.x - s.b.x) == 0) return s.a.y;
            return s.a.y + (s.b.y - s.a.y) * (x - s.a.x) / (s.b.x - s.a.x);
        };
        return dcmp(yvalSegment(p) - yvalSegment(q)) < 0;
    };

    multiset<int, decltype(comp)> st(comp);
    typedef decltype(st)::iterator iter;
    auto prev = [&st](iter it) {
        return it == st.begin() ? st.end() : --it;
    };
    auto next = [&st](iter it) {
        return it == st.end() ? st.end() : ++it;
    };

    vector<iter> pos(v.size());
    for(auto &cur : ev) {
        int id = cur.id;
        if(cur.tp == 1) {
            iter nxt = st.lower_bound(id);
            iter pre = prev(nxt);
            if(pre != st.end() && segmentsIntersect(v[*pre], v[id]))   return {*pre, id};
            if(nxt != st.end() && segmentsIntersect(v[*nxt], v[id]))   return {*nxt, id};
            pos[id] = st.insert(nxt, id);
        }
        else {
            iter nxt = next(pos[id]);
            iter pre = prev(pos[id]);
            if(pre != st.end() && nxt != st.end() && segmentsIntersect(v[*pre], v[*nxt]))
                return {*pre, *nxt};
            st.erase(pos[id]);
        }
    }
    return {-1, -1};
}\end{lstlisting}
\subsection{Linear}
\begin{lstlisting}
// returns true if point p is on segment s
bool onSegment(Point p, Segment s) {
    return dcmp(cross(s.a - p, s.b - p)) == 0 && dcmp(dot(s.a - p, s.b - p)) <= 0;
}

// returns true if segment p && q touch or intersect
bool segmentsIntersect(Segment p, Segment q) {
    if(onSegment(p.a, q) || onSegment(p.b, q)) return true;
    if(onSegment(q.a, p) || onSegment(q.b, p)) return true;

    Ti c1 = cross(p.b - p.a, q.a - p.a);
    Ti c2 = cross(p.b - p.a, q.b - p.a);
    Ti c3 = cross(q.b - q.a, p.a - q.a);
    Ti c4 = cross(q.b - q.a, p.b - q.a);
    return dcmp(c1) * dcmp(c2) < 0 && dcmp(c3) * dcmp(c4) < 0;
}

bool linesParallel(Line p, Line q) {
    return dcmp(cross(p.b - p.a, q.b - q.a)) == 0;
}

// lines are represented as a ray from a point: (point, vector)
// returns false if two lines (p, v) && (q, w) are parallel or collinear
// true otherwise, intersection point is stored at o via reference
bool lineLineIntersection(Point p, Point v, Point q, Point w, Point& o) {
    static_assert(is_same<Tf, Ti>::value);
    if(dcmp(cross(v, w)) == 0) return false;
    Point u = p - q;
    o = p + v * (cross(w,u)/cross(v,w));
    return true;
}

// returns false if two lines p && q are parallel or collinear
// true otherwise, intersection point is stored at o via reference
bool lineLineIntersection(Line p, Line q, Point& o) {
    return lineLineIntersection(p.a, p.b - p.a, q.a, q.b - q.a, o);
}

// returns the distance from point a to line l
Tf distancePointLine(Point p, Line l) {
    return abs(cross(l.b - l.a, p - l.a) / length(l.b - l.a));
}

// returns the shortest distance from point a to segment s
Tf distancePointSegment(Point p, Segment s) {
    if(s.a == s.b) return length(p - s.a);
    Point v1 = s.b - s.a, v2 = p - s.a, v3 = p - s.b;
    if(dcmp(dot(v1, v2)) < 0)       return length(v2);
    else if(dcmp(dot(v1, v3)) > 0)  return length(v3);
    else return abs(cross(v1, v2) / length(v1));
}

// returns the shortest distance from segment p to segment q
Tf distanceSegmentSegment(Segment p, Segment q) {
    if(segmentsIntersect(p, q)) return 0;
    Tf ans = distancePointSegment(p.a, q);
    ans = min(ans, distancePointSegment(p.b, q));
    ans = min(ans, distancePointSegment(q.a, p));
    ans = min(ans, distancePointSegment(q.b, p));
    return ans;
}

// returns the projection of point p on line l
Point projectPointLine(Point p, Line l) {
    static_assert(is_same<Tf, Ti>::value);
    Point v = l.b - l.a;
    return l.a + v * ((Tf) dot(v, p - l.a) / dot(v, v));
}\end{lstlisting}
\subsection{Point Rotation Trick}
\begin{lstlisting}
/// you may define the processor function in this namespace
/// instead of passing as an argument; testing shows function
/// defined using lambda and passed as argument performs better
/// tested on:
/// constant width strip - https://codeforces.com/gym/100016/problem/I
/// constant area triangle - https://codeforces.com/contest/1019/problem/D
/// smallest area quadrilateral - https://codingcompetitions.withgoogle.com/codejamio/round/000000000019ff03/00000000001b5e89
/// disjoint triangles count - https://codeforces.com/contest/1025/problem/F
/// smallest and largest triangle - http://serjudging.vanb.org/?p=561
typedef pair< int , int >PII;
void performTrick(vector< Point >pts, const function<void(const vector< Point >&, int)> &processor) {
    int n = pts.size();
    sort(pts.begin(), pts.end());
    vector< int >position(n);
    vector< PII >segments;
    segments.reserve((n*(n-1))/2);
    for (int i = 0; i < n; i++) {
        position[i] = i;
        for (int j = i+1; j < n; j++) {
            segments.emplace_back(i, j);
        }
    }
    assert(segments.capacity() == segments.size());
    sort(segments.begin(), segments.end(), [&](PII p, PII q) {
        Ti prod = cross(pts[p.second]-pts[p.first], pts[q.second]-pts[q.first]);
        if (prod != 0) return prod > 0;
        return p < q;
    });
    for (PII seg : segments) {
        int i = position[seg.first];
        assert(position[seg.second] == i+1);
        processor(pts, i);
        swap(pts[i], pts[i+1]);
        swap(position[seg.first], position[seg.second]);
    }
}\end{lstlisting}
\subsection{Point}
\begin{lstlisting}
typedef double Tf;
typedef double Ti;            /// use long long for exactness
const Tf PI = acos(-1), EPS = 1e-9;
int dcmp(Tf x) { return abs(x) < EPS ? 0 : (x<0 ? -1 : 1);}
 
struct Point {
    Ti x, y;
    Point(Ti x = 0, Ti y = 0) : x(x), y(y) {}
 
    Point operator + (const Point& u) const { return Point(x + u.x, y + u.y); }
    Point operator - (const Point& u) const { return Point(x - u.x, y - u.y); }
    Point operator * (const long long u) const { return Point(x * u, y * u); }
    Point operator * (const Tf u) const { return Point(x * u, y * u); }
    Point operator / (const Tf u) const { return Point(x / u, y / u); }
 
    bool operator == (const Point& u) const { return dcmp(x - u.x) == 0 && dcmp(y - u.y) == 0; }
    bool operator != (const Point& u) const { return !(*this == u); }
    bool operator < (const Point& u) const { return dcmp(x - u.x) < 0 || (dcmp(x - u.x) == 0 && dcmp(y - u.y) < 0); }
    friend istream &operator >> (istream &is, Point &p) { return is >> p.x >> p.y; }
    friend ostream &operator << (ostream &os, const Point &p) { return os << p.x << " " << p.y; }
};
 
Ti dot(Point a, Point b) { return a.x * b.x + a.y * b.y; }
Ti cross(Point a, Point b) { return a.x * b.y - a.y * b.x; }
Tf length(Point a) { return sqrt(dot(a, a)); }
Ti sqLength(Point a) { return dot(a, a); }
Tf distance(Point a, Point b) {return length(a-b);}
Tf angle(Point u) { return atan2(u.y, u.x); }
 
// returns angle between oa, ob in (-PI, PI]
Tf angleBetween(Point a, Point b) {
    Tf ans = angle(b) - angle(a);
    return ans <= -PI ? ans + 2*PI : (ans > PI ? ans - 2*PI : ans);
}
 
// Rotate a ccw by rad radians
Point rotate(Point a, Tf rad) {
    static_assert(is_same<Tf, Ti>::value);
    return Point(a.x * cos(rad) - a.y * sin(rad), a.x * sin(rad) + a.y * cos(rad));
}
 
// rotate a ccw by angle th with cos(th) = co && sin(th) = si
Point rotatePrecise(Point a, Tf co, Tf si) {
    static_assert(is_same<Tf, Ti>::value);
    return Point(a.x * co - a.y * si, a.y * co + a.x * si);
}
 
Point rotate90(Point a) { return Point(-a.y, a.x); }
 
// scales vector a by s such that length of a becomes s
Point scale(Point a, Tf s) {
    static_assert(is_same<Tf, Ti>::value);
    return a / length(a) * s;
}
 
// returns an unit vector perpendicular to vector a
Point normal(Point a) {
    static_assert(is_same<Tf, Ti>::value);
    Tf l = length(a);
    return Point(-a.y / l, a.x / l);
}
 
// returns 1 if c is left of ab, 0 if on ab && -1 if right of ab
int orient(Point a, Point b, Point c) {
    return dcmp(cross(b - a, c - a));
}
 
///Use as sort(v.begin(), v.end(), polarComp(O, dir))
///Polar comparator around O starting at direction dir
struct polarComp {
    Point O, dir;
    polarComp(Point O = Point(0, 0), Point dir = Point(1, 0))
        : O(O), dir(dir) {}
    bool half(Point p) {
        return dcmp(cross(dir, p)) < 0 || (dcmp(cross(dir, p)) == 0 && dcmp(dot(dir, p)) > 0);
    }
    bool operator()(Point p, Point q) {
        return make_tuple(half(p), 0) < make_tuple(half(q), cross(p, q));
    }
};
struct Segment {
    Point a, b;
    Segment(Point aa, Point bb) : a(aa), b(bb) {}
};
typedef Segment Line;
 
struct Circle {
    Point o;
    Tf r;
    Circle(Point o = Point(0, 0), Tf r = 0) : o(o), r(r) {}
 
    // returns true if point p is in || on the circle
    bool contains(Point p) {
        return dcmp(sqLength(p - o) - r * r) <= 0;
    }
 
    // returns a point on the circle rad radians away from +X CCW
    Point point(Tf rad) {
        static_assert(is_same<Tf, Ti>::value);
        return Point(o.x + cos(rad) * r, o.y + sin(rad) * r);
    }
 
    // area of a circular sector with central angle rad
    Tf area(Tf rad = PI + PI) { return rad * r * r / 2; }
 
    // area of the circular sector cut by a chord with central angle alpha
    Tf sector(Tf alpha) { return r * r * 0.5 * (alpha - sin(alpha)); }
};\end{lstlisting}
\subsection{Polygon}
\begin{lstlisting}
typedef vector<Point> Polygon;
// removes redundant colinear points
// polygon can't be all colinear points
Polygon RemoveCollinear(const Polygon& poly) {
    Polygon ret;
    int n = poly.size();
    for(int i = 0; i < n; i++) {
        Point a = poly[i];
        Point b = poly[(i + 1) % n];
        Point c = poly[(i + 2) % n];
        if(dcmp(cross(b-a, c-b)) != 0 && (ret.empty() || b != ret.back()))
            ret.push_back(b);
    }
    return ret;
}

// returns the signed area of polygon p of n vertices
Tf signedPolygonArea(const Polygon &p) {
    Tf ret = 0;
    for(int i = 0; i < (int) p.size() - 1; i++)
        ret += cross(p[i]-p[0],  p[i+1]-p[0]);
    return ret / 2;
}

// given a polygon p of n vertices, generates the convex hull in in CCW
// Tested on https://acm.timus.ru/problem.aspx?space=1&num=1185
// Caution: when all points are colinear AND removeRedundant == false
// output will be contain duplicate points (from upper hull) at back
Polygon convexHull(Polygon p, bool removeRedundant) {
    int check = removeRedundant ? 0 : -1;
    sort(p.begin(), p.end());
    p.erase(unique(p.begin(), p.end()), p.end());

    int n = p.size();
    Polygon ch(n+n);
    int m = 0;      // preparing lower hull
    for(int i = 0; i < n; i++) {
        while(m > 1 && dcmp(cross(ch[m - 1] - ch[m - 2], p[i] - ch[m - 1])) <= check) m--;
        ch[m++] = p[i];
    }
    int k = m;      // preparing upper hull
    for(int i = n - 2; i >= 0; i--) {
        while(m > k && dcmp(cross(ch[m - 1] - ch[m - 2], p[i] - ch[m - 2])) <= check) m--;
        ch[m++] = p[i];
    }
    if(n > 1) m--;
    ch.resize(m);
    return ch;
}

// Tested : https://www.spoj.com/problems/INOROUT
// returns inside = -1, on = 0, outside = 1
int pointInPolygon(const Polygon &p, Point o) {
    using Linear::onSegment;
    int wn = 0, n = p.size();
    for(int i = 0; i < n; i++) {
        int j = (i + 1) % n;
        if(onSegment(o, Segment(p[i], p[j])) || o == p[i]) return 0;
        int k = dcmp(cross(p[j] - p[i], o - p[i]));
        int d1 = dcmp(p[i].y - o.y);
        int d2 = dcmp(p[j].y - o.y);
        if(k > 0 && d1 <= 0 && d2 > 0) wn++;
        if(k < 0 && d2 <= 0 && d1 > 0) wn--;
    }
    return wn ? -1 : 1;
}

// Tested: Timus 1955, CF 598F
// Given a simple polygon p, and a line l, returns (x, y)
// x = longest segment of l in p, y = total length of l in p.
pair<Tf, Tf> linePolygonIntersection(Line l, const Polygon &p) {
    using Linear::lineLineIntersection;
    int n = p.size();
    vector<pair<Tf, int>> ev;
    for(int i=0; i<n; ++i) {
        Point a = p[i], b = p[(i+1)%n], z = p[(i-1+n)%n];
        int ora = orient(l.a, l.b, a), orb = orient(l.a, l.b, b), orz = orient(l.a, l.b, z);
        if(!ora) {
            Tf d = dot(a - l.a, l.b - l.a);
            if(orz && orb) {
                if(orz != orb) ev.emplace_back(d, 0);
                //else  // Point Touch
            }
            else if(orz) ev.emplace_back(d, orz);
            else if(orb) ev.emplace_back(d, orb);
        }
        else if(ora == -orb) {
            Point ins;
            lineLineIntersection(l, Line(a, b), ins);
            ev.emplace_back(dot(ins - l.a, l.b - l.a), 0);
        }
    }
    sort(ev.begin(), ev.end());

    Tf ans = 0, len = 0, last = 0, tot = 0;
    bool active = false;
    int sign = 0;
    for(auto &qq : ev) {
        int tp = qq.second;
        Tf d = qq.first;    /// current Segment is (last, d)
        if(sign) {          /// On Border
            len += d-last; tot += d-last;
            ans = max(ans, len);
            if(tp != sign) active = !active;
            sign = 0;
        }
        else {
            if(active) {  ///Strictly Inside
                len += d-last; tot += d-last;
                ans = max(ans, len);
            }
            if(tp == 0) active = !active;
            else sign = tp;
        }
        last = d;
        if(!active) len = 0;
    }
    ans /= length(l.b-l.a);
    tot /= length(l.b-l.a);
    return {ans, tot};
}\end{lstlisting}
\section{Graph}
\subsection{00}
\begin{lstlisting}
struct edge {
    int u, v;
    edge(int u = 0, int v = 0) : u(u), v(v) {}
    int to(int node){
        return u ^ v ^ node;
    }
};
struct graph {
    int n;
    vector<vector<int>> adj;
    vector <edge> edges;
    graph(int n = 0) : n(n), adj(n) {}
    void addEdge(int u, int v, bool dir = 1) {
        adj[u].push_back(edges.size());
        if(dir) adj[v].push_back(edges.size());
        edges.emplace_back(u, v);
    }
    int addNode() {
        adj.emplace_back();
        return n++;
    }
    edge &operator()(int idx) { return edges[idx]; }
    vector<int> &operator[](int u) { return adj[u]; }
};
\end{lstlisting}
\subsection{Block Cut Tree}
\begin{lstlisting}
vector < vector <int> > components;
vector <int> cutpoints, start, low;
vector <bool> is_cutpoint;
stack <int> st;
void find_cutpoints(int node, graph &G, int par = -1, int d = 0){
    low[node] = start[node] = d++;
    st.push(node);
    int cnt = 0;
    for(int e: G[node]) if(int to = G(e).to(node); to != par) {
        if(start[to] == -1){
            find_cutpoints(to, G, node, d+1);
            cnt++;
            if(low[to]  >= start[node]){
                is_cutpoint[node] = par != -1 or cnt > 1;
                components.push_back({node}); // starting a new block with the point
                while(st.top() != node)
                    components.back().push_back(st.top()), st.pop();    
            }
        }
        low[node] = min(low[node], low[to]);
    }
    
}
graph tree;
vector <int> id;
void init(graph &G) {
    int n = G.n;
    start.assign(n, -1), low.resize(n), is_cutpoint.resize(n), id.assign(n, -1);
    find_cutpoints(0, G);
    for (int u = 0; u < n; ++u)
        if (is_cutpoint[u]) 
            id[u] = tree.addNode();
    for (auto &comp : components) {
        int node = tree.addNode();
        for (int u : comp)
            if (!is_cutpoint[u]) id[u] = node;
            else tree.addEdge(node, id[u]);
    }
    if(id[0] == -1) // corner - 1
        id[0] = tree.addNode();
}
\end{lstlisting}
\subsection{Bridge Tree}
\begin{lstlisting}
vector<vector<int>> components;
vector<int> depth, low;
stack<int> st;
vector<int> id;
vector<edge> bridges;
graph tree;
void find_bridges(int node, graph &G, int par = -1, int d = 0) {
    low[node] = depth[node] = d;
    st.push(node);
    for (int id : G[node]) {
        int to = G(id).to(node);
        if (par != to) {
            if (depth[to] == -1) {
                find_bridges(to, G, node, d + 1);
                if (low[to] > depth[node]) {
                    bridges.emplace_back(node, to);
                    components.push_back({});
                    for (int x = -1; x != to; x = st.top(), st.pop())
                        components.back().push_back(st.top());
                }
            }
            low[node] = min(low[node], low[to]);
        }
    }
    if (par == -1) {
        components.push_back({});
        while (!st.empty()) components.back().push_back(st.top()), st.pop();
    }
}
graph &create_tree() {
    for (auto &comp : components) {
        int idx = tree.addNode();
        for (auto &e : comp) id[e] = idx;
    }
    for (auto &[l, r] : bridges) tree.addEdge(id[l], id[r]);
    return tree;
}
void init(graph &G) {
    int n = G.n;
    depth.assign(n, -1), id.assign(n, -1), low.resize(n);
    for (int i = 0; i < n; i++)
        if (depth[i] == -1) find_bridges(i, G);
}
\end{lstlisting}
\subsection{Centroid Decomposition}
\begin{lstlisting}
class Centroid_Decomposition {
    vector <bool> blocked;
    vector <int> CompSize;
    int CompDFS(tree &T, int node, int par) {
        CompSize[node] = 1;
        for(int &e: T[node]) if(e != par and !blocked[e])
            CompSize[node] += CompDFS(T, e, node);
        return CompSize[node];
    }
    int FindCentroid(tree &T, int node, int par, int sz) {
        for(int &e: T[node]) if(e != par and !blocked[e]) if(CompSize[e] > sz / 2)
            return FindCentroid(T, e, node, sz);
        return node;
    }
    pair <int,int> GetCentroid(tree &T, int entry) {
        int sz = CompDFS(T, entry, entry);
        return {FindCentroid(T, entry, entry, sz), sz};
    }
    c_vector <LL> left[2], right[2]; 
    int GMin, GMax;
    void dfs(tree &T, int node, int par, int Min, int Max, int sum) {
        if(blocked[node])
            return;
        right[Max < sum or Min > sum][sum]++;
        Max = max(Max, sum), Min = min(Min, sum);
        GMin = min(GMin, sum), GMax = max(GMax, sum);
        for(int i = 0; i < T[node].size(); i++) if(T[node][i] != par) {
            dfs(T, T[node][i], node, Min, Max, sum + T.col[node][i]);
        }
    }
    LL solve(tree &T, int c, int sz) {
        LL ans = 0;
        left[0].clear(-sz, sz), left[1].clear(-sz, sz);
        for(int i = 0; i < T[c].size(); i++) {
            GMin = 1, GMax = -1;
            dfs(T, T[c][i], c, GMin, GMax, T.col[c][i]);
            ans += right[0][0] + left[1][0] * right[1][0];
            for(int j:{0, 1} ) for(int k = GMin; k <= GMax; k++) {
                ans += right[j][k] * (left[0][-k] + (j == 0) * left[1][-k]);
            }
            for(int j:{0, 1} ) for(int k = GMin; k <= GMax; k++)  {
                left[j][k] += right[j][k];
                right[j][k] = 0;
            }
        }
        return ans;
    }
public:
    LL operator () (tree &T, int entry) {
        blocked.resize(T.n);
        CompSize.resize(T.n);
        for(int i:{0, 1})
            left[i].resize(2 * T.n + 5), right[i].resize(2 * T.n + 5);
        auto[c, sz] = GetCentroid(T, entry);
        LL ans = solve(T, c, sz);
        blocked[c] = true;
        for(int e: T[c]) if(!blocked[e])
            ans += (*this)(T, e);
        return ans;
    }   
};\end{lstlisting}
\subsection{Dinic Max Flow}
\begin{lstlisting}
using Ti = long long;
const Ti INF = 1LL << 60;
struct edge {
    int v, u;
    Ti cap, flow = 0;
    edge(int v, int u, Ti cap) : v(v), u(u), cap(cap) {}
};
const int N = 1e5 + 50;
vector <edge> edges;
vector <int> adj[N];
int m = 0, n;
int level[N], ptr[N];
queue <int> q;
bool bfs(int s, int t) {
    for (q.push(s), level[s] = 0; !q.empty(); q.pop()) {
        for (int id : adj[q.front()]) {
            auto &ed = edges[id];
            if (ed.cap - ed.flow > 0 and level[ed.u] == -1)
                level[ed.u] = level[ed.v] + 1, q.push(ed.u);
        }
    }
    return level[t] != -1;
}
Ti dfs(int v, Ti pushed, int t) {
    if (pushed == 0) return 0;
    if (v == t) return pushed;
    for (int& cid = ptr[v]; cid < adj[v].size(); cid++) {
        int id = adj[v][cid];
        auto &ed = edges[id];
        if (level[v] + 1 != level[ed.u] || ed.cap - ed.flow < 1) continue;
        Ti tr = dfs(ed.u, min(pushed, ed.cap - ed.flow), t);
        if (tr == 0) continue;
        ed.flow += tr;
        edges[id ^ 1].flow -= tr;
        return tr;
    }
    return 0;
}
void init(int nodes) {
    m = 0, n = nodes;
    for (int i = 0; i < n; i++) level[i] = -1, ptr[i] = 0, adj[i].clear();
}
void addEdge(int v, int u, Ti cap) {
    edges.emplace_back(v, u, cap), adj[v].push_back(m++);
    edges.emplace_back(u, v, 0), adj[u].push_back(m++);
}
Ti maxFlow(int s, int t) {
    Ti f = 0;
    for (auto &ed : edges) ed.flow = 0;
    for (; bfs(s, t); memset(level, -1, n * 4)) {
        for (memset(ptr, 0, n * 4); Ti pushed = dfs(s, INF, t); f += pushed);
    }
    return f;
}
\end{lstlisting}
\subsection{Euler Tour on Edge}
\begin{lstlisting}
// for simplicity, G[idx] contains the adjacency list of a node
// while G(e) is a reference to the e-th edge.
const int N = 2e5 + 5;
int in[N], out[N], fwd[N], bck[N];
int t = 0;
void dfs(graph &G, int node, int par) {
    out[node] = t; 
    for(int e: G[node]) {
        int v = G(e).to(node);
        if(v == par) continue;
        fwd[e] = t++;
        dfs(G, v, node);
        bck[e] = t++;
    }
    in[node] = t - 1;
}
void init(graph &G, int node) {
    t = 0;
    dfs(G, node, node);
}
\end{lstlisting}
\subsection{HLD}
\begin{lstlisting}
const int N = 1e6+7;
template <typename DT>
struct Segtree {
    vector<DT> tree, prob, a;
    Segtree(int n) {
        tree.resize(n * 4);
        prob.resize(n), a.resize(n);
    }
    void build(int u, int l, int r) {
        if (l == r) {
            tree[u] = a[l];
            return;
        }
        int mid = (l + r) / 2;
        build(u * 2, l, mid);
        build(u * 2 + 1, mid + 1, r);
        tree[u] = (tree[u * 2] + tree[u * 2 + 1]);
    }
    void propagate(int u) {
        prob[u * 2] += prob[u], tree[u * 2] += prob[u];
        prob[u * 2 + 1] += prob[u], tree[u * 2 + 1] += prob[u];
        prob[u] = 0;
    }
    void update(int u, int l, int r, int i, int j, int val) {
        if (r < i || l > j) return;
        if (l >= i && r <= j) {
            tree[u] = val;
            return;
        }
        int mid = (l + r) / 2;
        update(u * 2, l, mid, i, j, val);
        update(u * 2 + 1, mid + 1, r, i, j, val);
        tree[u] = (tree[u * 2] + tree[u * 2 + 1]);
    }
    DT query(int u, int l, int r, int i, int j) {
        if (l > j || r < i) return 0;
        if (l >= i && r <= j) return tree[u];
        int mid = (l + r) / 2;
        return (query(u * 2, l, mid, i, j) +  query(u * 2 + 1, mid + 1, r, i, j));
    }
};
Segtree<int>tree(N);
vector<int> adj[N];
int depth[N], par[N], pos[N];
int head[N], heavy[N], cnt;

int dfs(int u, int p) {
    int SZ = 1, mxsz = 0, heavyc;
    depth[u] = depth[p] + 1;

    for (auto v : adj[u]) {
        if (v == p) continue;
        par[v] = u;
        int subsz = dfs(v, u);
        if (subsz > mxsz) heavy[u] = v, mxsz = subsz;
        SZ += subsz;
    }
    return SZ;
}
void decompose(int u, int h) {
    head[u] = h, pos[u] = ++cnt;
    if(heavy[u]!=-1) decompose(heavy[u], h);

    for(int v : adj[u]) {
        if(v==par[u]) continue;
        if(v!=heavy[u]) decompose(v, v);
    }
}
int query(int a, int b) {
    int ret = 0;
    for(;head[a]!=head[b]; b=par[head[b]]){
       if(depth[head[a]]>depth[head[b]])  swap(a,b);
       ret += tree.query(1, 0, cnt, pos[head[b]], pos[b]);
    }

    if(depth[a]>depth[b])  swap(a,b);
    ret += tree.query(1,0,cnt,pos[a],pos[b]);
    return ret;
}\end{lstlisting}
\subsection{Hungarian}
\begin{lstlisting}
/**
    Hungarian algorithm for minimum weighted bipartite matching. (1-indexed)
    For max cost, negate cost matrix and negate output.
    Complexity: O(n^2 m). n must not be greater than m.

    Input: (n+1) x (m+1) cost matrix. (0th row and column are useless)
    Output: (ans, ml), where ml[i] = match for node i on the left.

    Source: upobir
*/
#include<bits/stdc++.h>
using namespace std;

template<typename T>
pair<T, vector<int>> Hungarian(const vector<vector<T>> &cost){
    const T INF = numeric_limits<T>::max();
    int n = cost.size()-1, m = cost[0].size()-1;
    vector<T> U(n+1), V(n+1);
    vector<int> mr(m+1), way(m+1), ml(n+1);

    for(int i = 1; i<=n; i++){
        mr[0] = i;
        int lastJ = 0;
        vector<T> minV(m+1, INF);
        vector<bool> used(m+1);
        do{
            used[lastJ] = true;
            int lastI = mr[lastJ], nextJ;
            T delta = INF;
            for(int j = 1; j<=m; j++){
                if(used[j]) continue;
                T diffCost = cost[lastI][j] - U[lastI] - V[j];
                if(diffCost < minV[j]) minV[j] = diffCost, way[j] = lastJ;
                if(minV[j] < delta) delta = minV[j], nextJ = j;
            }
            for(int j = 0; j<=m; j++){
                if(used[j]) U[mr[j]] += delta, V[j] -= delta;
                else        minV[j] -= delta;
            }
            lastJ = nextJ;
        } while(mr[lastJ] != 0);
        do{
            int prevJ = way[lastJ];
            mr[lastJ] = mr[prevJ];
            lastJ = prevJ;
        } while(lastJ != 0);
    }
    for (int i=1; i<=m; i++)    ml[mr[i]] = i;
    return {-V[0], ml} ;
}


int main() {
    ios::sync_with_stdio(0);
    cin.tie(0);

    int n;
    cin>>n;

    vector<vector<long long>> cost(n+1, vector<long long>(n+1));
    for (int i=1; i<=n; i++)
        for (int j=1; j<=n; j++)    cin>>cost[i][j];

    auto [ans, match] = Hungarian(cost);
    cout<<ans<<endl;
    for (int i=1; i<=n; i++)    cout<<match[i]-1<<" ";
}
\end{lstlisting}
\subsection{LCA In O(1)}
\begin{lstlisting}
/* 
  * LCA in O(1)
  * depth calculates weighted distance  
  * level calculates distance by number of edges
  * Preprocessing in NlongN
*/

#include <bits/stdc++.h>
using namespace std;

typedef long long LL;
typedef pair<int, int> PII;

const int N = 1e6 + 7;
const int L = 21;


namespace LCA {
LL depth[N];
int level[N];  

int st[N], en[N], LOG[N], par[N];
int a[N], id[N], table[L][N];

vector<PII> adj[N];
int n, root, Time, cur;

void init(int nodes, int root_) {
    n = nodes, root = root_, LOG[0] = LOG[1] = 0;
    for (int i = 2; i <= n; i++) LOG[i] = LOG[i >> 1] + 1;
    for (int i = 0; i <= n; i++) adj[i].clear();
}

void addEdge(int u, int v, int w) {
    adj[u].push_back(PII(v, w));
    adj[v].push_back(PII(u, w));
}

int lca(int u, int v) {
    if (en[u] > en[v]) swap(u, v);
    if (st[v] <= st[u] && en[u] <= en[v]) return v;

    int l = LOG[id[v] - id[u] + 1];
    int p1 = id[u], p2 = id[v] - (1 << l) + 1;
    int d1 = level[table[l][p1]], d2 = level[table[l][p2]];

    if (d1 < d2)  return par[table[l][p1]];
    else  return par[table[l][p2]];
}

LL dist(int u, int v) {
    int l = lca(u, v);
    return (depth[u] + depth[v] - (depth[l] * 2));
}

/* Euler tour */
void dfs(int u, int p) {
    st[u] = ++Time, par[u] = p;

    for (auto [v, w] : adj[u]) {
        if (v == p) continue;
        depth[v] = depth[u] + w;
        level[v] = level[u] + 1;
        dfs(v, u);
    }

    en[u] = ++Time;
    a[++cur] = u, id[u] = cur;
}

/* RMQ */
void pre() {
    cur = Time = 0, dfs(root, root);
    for (int i = 1; i <= n; i++) table[0][i] = a[i];

    for (int l = 0; l < L - 1; l++) {
        for (int i = 1; i <= n; i++) {
            table[l + 1][i] = table[l][i];

            bool C1 = (1 << l) + i <= n;
            bool C2 = level[table[l][i + (1 << l)]] < level[table[l][i]];

            if (C1 && C2) table[l + 1][i] = table[l][i + (1 << l)];
        }
    }
}

} 
/* namespace LCA */
//tested on kattis-greatestpair

using namespace LCA;

int main() {
    ios::sync_with_stdio(0);
    cin.tie(0);

}
\end{lstlisting}
\subsection{Min Cost Max Flow}
\begin{lstlisting}
mt19937 rnd(chrono::steady_clock::now().time_since_epoch().count());
const LL inf = 1e9;
struct edge {
    int v, rev;
    LL cap, cost, flow;
    edge() {}
    edge(int v, int rev, LL cap, LL cost)
        : v(v), rev(rev), cap(cap), cost(cost), flow(0) {}
};
struct mcmf {
    int src, sink, n;
    vector<int> par, idx, Q;
    vector<bool> inq;
    vector<LL> dis;
    vector<vector<edge>> g;
    mcmf() {}
    mcmf(int src, int sink, int n)
        : src(src), sink(sink), n(n), par(n), idx(n), inq(n), dis(n), g(n),
          Q(10000005) {}  // use Q(n) if not using random
    void add_edge(int u, int v, LL cap, LL cost, bool directed = true) {
        edge _u = edge(v, g[v].size(), cap, cost);
        edge _v = edge(u, g[u].size(), 0, -cost);
        g[u].pb(_u);
        g[v].pb(_v);
        if (!directed) add_edge(v, u, cap, cost, true);
    }
    bool spfa() {
        for (int i = 0; i < n; i++) {
            dis[i] = inf, inq[i] = false;
        }
        int f = 0, l = 0;
        dis[src] = 0, par[src] = -1, Q[l++] = src, inq[src] = true;
        while (f < l) {
            int u = Q[f++];
            for (int i = 0; i < g[u].size(); i++) {
                edge &e = g[u][i];
                if (e.cap <= e.flow) continue;
                if (dis[e.v] > dis[u] + e.cost) {
                    dis[e.v] = dis[u] + e.cost;
                    par[e.v] = u, idx[e.v] = i;
                    if (!inq[e.v]) inq[e.v] = true, Q[l++] = e.v;
                    // if (!inq[e.v]) {
                    //   inq[e.v] = true;
                    //   if (f && rnd() & 7) Q[--f] = e.v;
                    //   else Q[l++] = e.v;
                    // }
                }
            }
            inq[u] = false;
        }
        return (dis[sink] != inf);
    }
    pair<LL, LL> solve() {
        LL mincost = 0, maxflow = 0;
        while (spfa()) {
            LL bottleneck = inf;
            for (int u = par[sink], v = idx[sink]; u != -1;
                 v = idx[u], u = par[u]) {
                edge &e = g[u][v];
                bottleneck = min(bottleneck, e.cap - e.flow);
            }
            for (int u = par[sink], v = idx[sink]; u != -1;
                 v = idx[u], u = par[u]) {
                edge &e = g[u][v];
                e.flow += bottleneck;
                g[e.v][e.rev].flow -= bottleneck;
            }
            mincost += bottleneck * dis[sink], maxflow += bottleneck;
        }
        return make_pair(mincost, maxflow);
    }
};
// want to minimize cost and don't care about flow
// add edge from sink to dummy sink (cap = inf, cost = 0)
// add edge from source to sink (cap = inf, cost = 0)
// run mcmf, cost returned is the minimum cost
\end{lstlisting}
\subsection{SCC}
\begin{lstlisting}
typedef long long LL;
const LL N = 1e6 + 7;

bool vis[N];
vector<int> adj[N], adjr[N];
vector<int> order, component;
// tp = 0 ,finding topo order, tp = 1 , reverse edge traversal

void dfs(int u, int tp = 0) {
    vis[u] = true;
    if (tp) component.push_back(u);
    auto& ad = (tp ? adjr : adj);
    for (int v : ad[u])
        if (!vis[v]) dfs(v, tp);
    if (!tp) order.push_back(u);
}
int main() {
    for (int i = 1; i <= n; i++) {
        if (!vis[i]) dfs(i);
    }
    memset(vis, 0, sizeof vis);
    reverse(order.begin(), order.end());
    for (int i : order) {
        if (!vis[i]) {
            // one component is found
            dfs(i, 1), component.clear();
        }
    }
}

\end{lstlisting}
\subsection{StoerWanger}
\begin{lstlisting}
/* for finding the min cut of a graph without specifing the source and  the sink.
   all the edges are directed and no need to  make any edge bidirectional.
*/
const int N = 1407;
// O(n^3) but faster, 1 indexed

mt19937 rnd(chrono::steady_clock::now().time_since_epoch().count());
struct StoerWagner {
    int n, idx[N];
    LL G[N][N], dis[N];
    bool vis[N];
    const LL inf = 1e18;

    StoerWagner() {}
    StoerWagner(int _n) {
        n = _n;
        memset(G, 0, sizeof G);
    }
    void add_edge(int u, int v, LL w) {  // undirected edge, multiple edges are merged into one edge
        if (u != v) G[u][v] += w, G[v][u] += w;
    }

    LL solve() {
        LL ans = inf;
        for (int i = 0; i < n; ++i) idx[i] = i + 1;
        shuffle(idx, idx + n, rnd);

        while (n > 1) {
            int t = 1, s = 0;
            for (int i = 1; i < n; ++i) {
                dis[idx[i]] = G[idx[0]][idx[i]];
                if (dis[idx[i]] > dis[idx[t]]) t = i;
            }

            memset(vis, 0, sizeof vis);
            vis[idx[0]] = true;

            for (int i = 1; i < n; ++i) {
                if (i == n - 1) {
                    if (ans > dis[idx[t]])
                        ans = dis[idx[t]];  // idx[s] - idx[t] is in two halves of the  mincut
                    if (ans == 0) return 0;
                    for (int j = 0; j < n; ++j) {
                        G[idx[s]][idx[j]] += G[idx[j]][idx[t]];
                        G[idx[j]][idx[s]] += G[idx[j]][idx[t]];
                    }
                    idx[t] = idx[--n];
                }

                vis[idx[t]] = true;
                s = t, t = -1;

                for (int j = 1; j < n; ++j) {
                    if (!vis[idx[j]]) {
                        dis[idx[j]] += G[idx[s]][idx[j]];
                        if (t == -1 || dis[idx[t]] < dis[idx[j]]) t = j;
                    }
                }
            }
        }
        return ans;
    }
};\end{lstlisting}
\subsection{Tree Algo}
\begin{lstlisting}
struct tree {
    int n;
    vector <vector <int> > adj;
    inline vector<int>& operator[](int u) {
        return adj[u];
    }
    tree(int n = 0) : n(n), adj(n) {}
    void addEdge(int u, int v) {
        adj[u].push_back(v);
        adj[v].push_back(u);
    }
};
struct lca_table {
    tree &T;
    int n, LOG = 20;
    vector <vector <int>> anc;
    vector <int> level;

    void setupLifting(int node, int par) {
        for (int v : T[node]) if (v != par) {
                anc[v][0] = node, level[v] = level[node] + 1;
                for (int k = 1; k < LOG; k++)
                    anc[v][k] = anc[anc[v][k - 1]][k - 1];
                setupLifting(v, node);
            }
    }
    lca_table(tree &T, int root = 0): T(T), n(T.n) {
        LOG = 33 - __builtin_clz(n);
        anc.assign(n, vector <int> (LOG, root));
        level.resize(n);
        setupLifting(root, root);
    }
    int lca(int u, int v) {
        if (level[u] > level[v])
            swap(u, v);
        for (int k = LOG - 1; ~k; k--)
            if (level[u] + (1 << k) <= level[v])
                v = anc[v][k];
        if (u == v)
            return u;
        for (int k = LOG - 1; ~k; k--)
            if (anc[u][k] != anc[v][k])
                u = anc[u][k], v = anc[v][k];
        return anc[u][0];
    }
    int getAncestor(int node, int ht) {
        for (int k = 0; k < LOG; k++)
            if (ht & (1 << k))
                node = anc[node][k];
        return node;
    }
    int distance(int u, int v) {
        int g = lca(u, v);
        return level[u] + level[v] - 2 * level[g];
    }
};
struct euler_tour {
    int time = 0;
    tree &T;
    int n;
    vector <int> start, finish, level, par;
    euler_tour(tree &T, int root = 0) : T(T), n(T.n), start(n), finish(n), level(n), par(n) {
        time = 0;
        call(root);
    }
    void call(int node, int p = -1) {
        if (p != -1) level[node] = level[p] + 1;
        start[node] = time++;
        for (int e : T[node]) if (e != p)
                call(e, node);
        par[node] = p;
        finish[node] = time++;
    }
    bool isAncestor(int node, int par) {
        return start[par] <= start[node] and finish[par] >= finish[node];
    }
    int subtreeSize(int node) {
        return finish[node] - start[node] + 1 >> 1;
    }
};
tree virtual_tree(vector <int> &nodes, lca_table &table, euler_tour &tour) {
    sort(nodes.begin(), nodes.end(), [&](int x, int y) {
        return tour.start[x] < tour.start[y];
    });
    int n = nodes.size();
    for (int i = 0; i + 1 < n; i++)
        nodes.push_back(table.lca(nodes[i], nodes[i + 1]));
    sort(nodes.begin(), nodes.end());
    nodes.erase(unique(nodes.begin(), nodes.end()), nodes.end());
    sort(nodes.begin(), nodes.end(), [&](int x, int y) {
        return tour.start[x] < tour.start[y];
    });
    n = nodes.size();
    stack <int> st;
    st.push(0);
    tree ans(n);
    for (int i = 1; i < n; i++) {
        while (!tour.isAncestor(nodes[i], nodes[st.top()])) st.pop();
        ans.addEdge(st.top(), i);
        st.push(i);
    }
    return ans;
}
set <int> getCenters(tree &T) {
    int n = T.n;
    vector <int> deg(n), q;
    set <int> s;
    for (int i = 0; i < n; i++) {
        deg[i] = T[i].size();
        if (deg[i] == 1)
            q.push_back(i);
        s.insert(i);
    }
    for (vector <int> t ; s.size() > 2; q = t) {
        for (auto x : q) {
            for (auto e : T[x]) 
                if (--deg[e] == 1)
                    t.push_back(e);
            s.erase(x);
        }
    }
    return s;
}
bool check(tree &T) {
    for (int i = 0; i < T.n; i++)
        if (T[i].size() > 2) return false;
    return true;
}
\end{lstlisting}
\subsection{Tree Isomorphism}
\begin{lstlisting}
mp["01"] = 1;
ind = 1;
int dfs(int u, int p) {
  int cnt = 0;
  vector<int>vs;
  for (auto v : g1[u]) {
    if (v != p) {
      int got = dfs(v, u);
      vs.pb(got);
      cnt++;
    }
  }
  if (!cnt) return 1;

  sort(vs.begin(), vs.end());
  string s = "0";
  for (auto i : vs) s += to_string(i);
  vs.clear();
  s.pb('1');
  if (mp.find(s) == mp.end()) mp[s] = ++ind;
  int ret = mp[s];
  return ret;
}
\end{lstlisting}
\section{Math}
\subsection{Adaptive Simpsons}
\begin{lstlisting}
/*
    For finding the length of an arc in a range
    L = integrate(ds) from start to end of range
    where ds = sqrt(1+(d/dy(x))^2)dy
*/
const double SIMPSON_TERMINAL_EPS = 1e-12;
/// Function whose integration is to be calculated
double F(double x);
double simpson(double minx, double maxx)
{
    return (maxx - minx) / 6 * (F(minx) + 4 * F((minx + maxx) / 2.) + F(maxx));
}
double adaptive_simpson(double minx, double maxx, double c, double EPS)
{
//    if(maxx - minx < SIMPSON_TERMINAL_EPS) return 0;

    double midx = (minx + maxx) / 2;
    double a = simpson(minx, midx);
    double b = simpson(midx, maxx);

    if(fabs(a + b - c) < 15 * EPS) return a + b + (a + b - c) / 15.0;

    return adaptive_simpson(minx, midx, a, EPS / 2.) + adaptive_simpson(midx, maxx, b, EPS / 2.);
}
double adaptive_simpson(double minx, double maxx, double EPS)
{
    return adaptive_simpson(minx, maxx, simpson(minx, maxx, i), EPS);
}

\end{lstlisting}
\subsection{Berlekamp Massey}
\begin{lstlisting}
struct berlekamp_massey { // for linear recursion
  typedef long long LL;
  static const int SZ = 2e5 + 5;
  static const int MOD = 1e9 + 7; /// mod must be a prime
  LL m , a[SZ] , h[SZ] , t_[SZ] , s[SZ] , t[SZ];
  // bigmod goes here
  inline vector <LL> BM( vector <LL> &x ) {
    LL lf , ld;
    vector <LL> ls , cur;
    for ( int i = 0; i < int(x.size()); ++i ) {
      LL t = 0;
      for ( int j = 0; j < int(cur.size()); ++j ) t = (t + x[i - j - 1] * cur[j]) % MOD;
      if ( (t - x[i]) % MOD == 0 ) continue;
      if ( !cur.size() ) {
        cur.resize( i + 1 );
        lf = i; ld = (t - x[i]) % MOD;
        continue;
      }
      LL k = -(x[i] - t) * bigmod( ld , MOD - 2 , MOD ) % MOD;
      vector <LL> c(i - lf - 1);
      c.push_back( k );
      for ( int j = 0; j < int(ls.size()); ++j ) c.push_back(-ls[j] * k % MOD);
      if ( c.size() < cur.size() ) c.resize( cur.size() );
      for ( int j = 0; j < int(cur.size()); ++j ) c[j] = (c[j] + cur[j]) % MOD;
      if (i - lf + (int)ls.size() >= (int)cur.size() ) ls = cur, lf = i, ld = (t - x[i]) % MOD;
      cur = c;
    }
    for ( int i = 0; i < int(cur.size()); ++i ) cur[i] = (cur[i] % MOD + MOD) % MOD;
    return cur;
  }
  inline void mull( LL *p , LL *q ) {
    for ( int i = 0; i < m + m; ++i ) t_[i] = 0;
    for ( int i = 0; i < m; ++i ) if ( p[i] )
        for ( int j = 0; j < m; ++j ) t_[i + j] = (t_[i + j] + p[i] * q[j]) % MOD;
    for ( int i = m + m - 1; i >= m; --i ) if ( t_[i] )
        for ( int j = m - 1; ~j; --j ) t_[i - j - 1] = (t_[i - j - 1] + t_[i] * h[j]) % MOD;
    for ( int i = 0; i < m; ++i ) p[i] = t_[i];
  }
  inline LL calc( LL K ) {
    for ( int i = m; ~i; --i ) s[i] = t[i] = 0;
    s[0] = 1; if ( m != 1 ) t[1] = 1; else t[0] = h[0];
    while ( K ) {
      if ( K & 1 ) mull( s , t );
      mull( t , t ); K >>= 1;
    }
    LL su = 0;
    for ( int i = 0; i < m; ++i ) su = (su + s[i] * a[i]) % MOD;
    return (su % MOD + MOD) % MOD;
  }
  /// already calculated upto k , now calculate upto n.
  inline vector <LL> process( vector <LL> &x , int n , int k ) {
    auto re = BM( x );
    x.resize( n + 1 );
    for ( int i = k + 1; i <= n; i++ ) {
      for ( int j = 0; j < re.size(); j++ ) {
        x[i] += 1LL * x[i - j - 1] % MOD * re[j] % MOD; x[i] %= MOD;
      }
    }
    return x;
  }
  inline LL work( vector <LL> &x , LL n ) {
    if ( n < int(x.size()) ) return x[n] % MOD;
    vector <LL> v = BM( x ); m = v.size(); if ( !m ) return 0;
    for ( int i = 0; i < m; ++i ) h[i] = v[i], a[i] = x[i];
    return calc( n ) % MOD;
  }
} rec;
vector <LL> v;
void solve() {
  int n;
  cin >> n;
  cout << rec.work(v, n - 1) << endl;
}

\end{lstlisting}
\subsection{Chinese Remainder Theorem}
\begin{lstlisting}
// given a, b will find solutions for
// ax + by = 1
tuple <LL,LL,LL> EGCD(LL a, LL b){
    if(b == 0) return {1, 0, a};
    else{
        auto [x,y,g] = EGCD(b, a%b);
        return {y, x - a/b*y,g};
    }
}
// given modulo equations, will apply CRT
PLL CRT(vector <PLL> &v){
    LL V = 0, M = 1;
    for(auto &[v, m]:v){ //value % mod
        auto [x, y, g] = EGCD(M, m);
        if((v - V) % g != 0)
            return {-1, 0};
        V += x * (v - V) / g % (m / g) * M, M *= m / g;
        V = (V % M + M) % M;
    }
    return make_pair(V, M);
}
\end{lstlisting}
\subsection{Combi}
\begin{lstlisting}
const int N = 2e5+5;
const int mod = 1e9+7;

namespace com{
    array <int, N+1> fact, inv, inv_fact;
    void init(){
        fact[0] = inv_fact[0] = 1;
        for(int i = 1; i <= N; i++){
            inv[i] = i == 1 ? 1 : (LL) inv[i - mod%i] * (mod/i + 1) % mod;
            fact[i] = (LL) fact[i-1] * i % mod;
            inv_fact[i] = (LL) inv_fact[i-1] * inv[i] % mod;
        }
    }
    LL C(int n,int r){
        return (r < 0 or r > n) ? 0 : (LL) fact[n]*inv_fact[r] % mod * inv_fact[n-r] % mod;
    }
}
\end{lstlisting}
\subsection{FFT}
\begin{lstlisting}
using CD = complex<double>;
typedef long long LL;
const double PI = acos(-1.0L);

int N;
vector<int> perm;
vector<CD> wp[2];
void precalculate(int n) {
    assert((n & (n - 1)) == 0), N = n;
    perm = vector<int>(N, 0);
    for (int k = 1; k < N; k <<= 1) {
        for (int i = 0; i < k; i++) {
            perm[i] <<= 1;
            perm[i + k] = 1 + perm[i];
        }
    }
    wp[0] = wp[1] = vector<CD>(N);
    for (int i = 0; i < N; i++) {
        wp[0][i] = CD(cos(2 * PI * i / N), sin(2 * PI * i / N));
        wp[1][i] = CD(cos(2 * PI * i / N), -sin(2 * PI * i / N));
    }
}
void fft(vector<CD> &v, bool invert = false) {
    if (v.size() != perm.size()) precalculate(v.size());
    for (int i = 0; i < N; i++)
        if (i < perm[i]) swap(v[i], v[perm[i]]);
    for (int len = 2; len <= N; len *= 2) {
        for (int i = 0, d = N / len; i < N; i += len) {
            for (int j = 0, idx = 0; j < len / 2; j++, idx += d) {
                CD x = v[i + j];
                CD y = wp[invert][idx] * v[i + j + len / 2];
                v[i + j] = x + y;
                v[i + j + len / 2] = x - y;
            }
        }
    }
    if (invert) {
        for (int i = 0; i < N; i++) v[i] /= N;
    }
}
void pairfft(vector<CD> &a, vector<CD> &b, bool invert = false) {
    int N = a.size();
    vector<CD> p(N);
    for (int i = 0; i < N; i++) p[i] = a[i] + b[i] * CD(0, 1);
    fft(p, invert);
    p.push_back(p[0]);
    for (int i = 0; i < N; i++) {
        if (invert) {
            a[i] = CD(p[i].real(), 0);
            b[i] = CD(p[i].imag(), 0);
        } else {
            a[i] = (p[i] + conj(p[N - i])) * CD(0.5, 0);
            b[i] = (p[i] - conj(p[N - i])) * CD(0, -0.5);
        }
    }
}
vector<LL> multiply(const vector<LL> &a, const vector<LL> &b) {
    int n = 1;
    while (n < a.size() + b.size()) n <<= 1;
    vector<CD> fa(a.begin(), a.end()), fb(b.begin(), b.end());
    fa.resize(n);
    fb.resize(n);
    //        fft(fa); fft(fb);
    pairfft(fa, fb);
    for (int i = 0; i < n; i++) fa[i] = fa[i] * fb[i];
    fft(fa, true);
    vector<LL> ans(n);
    for (int i = 0; i < n; i++) ans[i] = round(fa[i].real());
    return ans;
}
const int M = 1e9 + 7, B = sqrt(M) + 1;
vector<LL> anyMod(const vector<LL> &a, const vector<LL> &b) {
    int n = 1;
    while (n < a.size() + b.size()) n <<= 1;
    vector<CD> al(n), ar(n), bl(n), br(n);
    for (int i = 0; i < a.size(); i++)
        al[i] = a[i] % M / B, ar[i] = a[i] % M % B;
    for (int i = 0; i < b.size(); i++)
        bl[i] = b[i] % M / B, br[i] = b[i] % M % B;
    pairfft(al, ar);
    pairfft(bl, br);
    //        fft(al); fft(ar); fft(bl); fft(br);
    for (int i = 0; i < n; i++) {
        CD ll = (al[i] * bl[i]), lr = (al[i] * br[i]);
        CD rl = (ar[i] * bl[i]), rr = (ar[i] * br[i]);
        al[i] = ll;
        ar[i] = lr;
        bl[i] = rl;
        br[i] = rr;
    }
    pairfft(al, ar, true);
    pairfft(bl, br, true);
    //        fft(al, true); fft(ar, true); fft(bl, true); fft(br, true);
    vector<LL> ans(n);
    for (int i = 0; i < n; i++) {
        LL right = round(br[i].real()), left = round(al[i].real());
        ;
        LL mid = round(round(bl[i].real()) + round(ar[i].real()));
        ans[i] = ((left % M) * B * B + (mid % M) * B + right) % M;
    }
    return ans;
}\end{lstlisting}
\subsection{Fractional Binary Search}
\begin{lstlisting}
/**
Given a function f and n, finds the smallest fraction p / q in [0, 1] or [0,n]
such that f(p / q) is true, and p, q <= n.
Time: O(log(n))
**/
struct frac { long long p, q; };
bool f(frac x) {
 return 6 + 8 * x.p >= 17 * x.q + 12;
}
frac fracBS(long long n) {
  bool dir = 1, A = 1, B = 1;
  frac lo{0, 1}, hi{1, 0}; // Set hi to 1/0 to search within [0, n] and {1, 1} to search within [0, 1]
  if (f(lo)) return lo;
  assert(f(hi)); //checking if any solution exists or not
  while (A || B) {
    long long adv = 0, step = 1; // move hi if dir, else lo
    for (int si = 0; step; (step *= 2) >>= si) {
      adv += step;
      frac mid{lo.p * adv + hi.p, lo.q * adv + hi.q};
      if (abs(mid.p) > n || mid.q > n || dir == !f(mid)) {
        adv -= step; si = 2;
      } 
    }
    hi.p += lo.p * adv;
    hi.q += lo.q * adv;
    dir = !dir;
    swap(lo, hi);
    A = B; B = !!adv;
  }
  return dir ? hi : lo;
}
\end{lstlisting}
\subsection{Gaussian Elimination}
\begin{lstlisting}
double gaussian_elimination(int row, int col) {
    int basis[30];
    for (int j = 0; j < row; j++) {
        MAT[j][j + col] = 1;
    }
    memset(basis, -1, sizeof basis);
    double det = 1;
    for (int i = 0; i < col; i++) {
        for (int p = 0; p < row ; p++) {
            for (int q = 0; q < col; q++)
                cout << MAT[p][q] << ' ';
            cout << '\n';
        }
        int x = -1;
        for (int k = 0; k < row; k++) {
            if (abs(MAT[k][i]) > eps and basis[k] == -1) {
                x = k, det *= MAT[k][i], basis[x] = i;
                break;
            }
        }
        if (x < 0) continue;
        for (int j = 0; j < col; j++)
            if (j != i)  for (int k = 0; k < row; k++) if (k != x)
                        MAT[k][j] -= (MAT[k][i] * MAT[x][j]) / MAT[x][i];
        for (int k = 0; k < col; k++) if (k != i)
                MAT[x][k] /= MAT[x][i];
        for (int j = 0; j < row; j++)
            MAT[j][i] = (j == i);
    }
    for (int i = 0; i < row ; i++) {
        for (int j = 0; j < col; j++)
            cout << MAT[i][j] << ' ';
        cout << '\n';
    }
    for (int k = 0; k < row; k++)
        if (basis[k] == -1)
            return 0;
    return det;
}\end{lstlisting}
\subsection{Green Hackenbush on Trie}
\begin{lstlisting}
int trie[40 * MAX][26];
int XOR[40 * MAX][26];
int valu[40 * MAX];
int node = 1;

int add(string s) {
  int now = 1;
  stack<int>st;
  for (int i = 0; i < s.size(); i++) {
    int c = s[i] - 'a';
    if (!trie[now][c]) trie[now][c] = ++node;
    st.push(now);
    now = trie[now][c];
  }

  int nxt = now;
  int nxt_val = 0;
  for (int i = 0; i < 26; i++) nxt_val ^= XOR[now][i];
  while (!st.empty()) {
    now = st.top();
    st.pop();
    int val = 0;
    for (int i = 0; i < 26; i++) {
      if (trie[now][i] == nxt) {
        XOR[now][i] = nxt_val + 1;
      }
      val ^= XOR[now][i];
    }
    nxt_val = val;
    nxt = now;
  }
  return nxt_val;
}
\end{lstlisting}
\subsection{Lagrange}
\begin{lstlisting}
// p is a polynomial with n points.
// p(0), p(1), p(2), ... p(n-1) are given.
// Find p(x).

LL Lagrange(vector<LL> &p, LL x)
{
    LL n = p.size(), L, i, ret;
 
    if(x < n)
        return p[x];
 
    L = 1;
    for(i = 1; i < n; i++)
    {
        L = (L * (x - i)) % MOD;
        L = (L * bigmod(MOD - i, MOD - 2)) % MOD;
    }
 
    ret = (L * p[0]) % MOD;
 
    for(i = 1; i < n; i++)
    {
        L = (L*(x - i + 1)) % MOD;
        L = (L*bigmod(x - i, MOD-2)) % MOD;
        
        L = (L*bigmod(i, MOD-2)) % MOD;
        L = (L*(MOD+i-n)) % MOD;
 
        ret = (ret + L*p[i]) % MOD;
    }
 
    return ret;
}\end{lstlisting}
\subsection{Linear Sieve}
\begin{lstlisting}
const int N = 1e7;
vector <int> primes;
int spf[N+5], phi[N+5], NOD[N+5], cnt[N+5], POW[N+5]; 
bool prime[N+5];
int SOD[N+5];
void init(){
    fill(prime+2, prime+N+1, 1);
    SOD[1] = NOD[1] = phi[1] = spf[1] = 1;
    for(LL i=2;i<=N;i++){
        if(prime[i]) {
            primes.push_back(i), spf[i] = i;
            phi[i] = i-1;
            NOD[i] = 2, cnt[i] = 1;
            SOD[i] = i+1, POW[i] = i;
        }
        for(auto p:primes){
            if(p*i>N or p > spf[i]) break;
            prime[p*i] = false, spf[p*i] = p;
            if(i%p == 0){
                phi[p*i]=p*phi[i];
                NOD[p*i]=NOD[i]/(cnt[i]+1)*(cnt[i]+2), cnt[p*i]=cnt[i]+1;
                SOD[p*i]=SOD[i]/SOD[POW[i]]*(SOD[POW[i]]+p*POW[i]),POW[p*i]=p*POW[i];
                break;
            } else {
                phi[p*i]=phi[p]*phi[i];
                NOD[p*i]=NOD[p]*NOD[i], cnt[p*i]=1;
                SOD[p*i]=SOD[p]*SOD[i], POW[p*i]=p;
            }

        }
    }
}

\end{lstlisting}
\subsection{Matrix Exponentiation}
\begin{lstlisting}
typedef vector<vector<LL>> Mat;

Mat Mul(Mat A, Mat B)
{
    Mat ret(A.size(), vector<LL>(B[0].size()));
    LL i, j, k;

    for(i = 0; i < ret.size(); i++)
    {
        for(j = 0; j < ret[0].size(); j++)
        {
            for(k = 0; k < A[0].size(); k++)
                ret[i][j] = (ret[i][j] + (A[i][k]*B[k][j])%MOD)%MOD;
        }
    }

    return ret;
}

Mat Pow(Mat A, LL p)
{
    Mat ret(A.size(), vector<LL>(A[0].size()));

    for(LL i = 0; i < ret.size(); i++)
        ret[i][i] = 1;

    while(p)
    {
        if(p&1)
            ret = Mul(ret, A);
        A = Mul(A, A);
        p >>= 1;
    }
    return ret;
}
\end{lstlisting}
\subsection{Mobius Function}
\begin{lstlisting}
const int N = 1e6 + 5;
int mob[N];

void mobius() {
    memset(mob, -1, sizeof mob);
    mob[1] = 1;
    for (int i = 2; i < N; i++) if (mob[i]){
        for (int j = i + i; j < N; j += i) 
            mob[j] -= mob[i];
    }
}

\end{lstlisting}
\subsection{NTT}
\begin{lstlisting}
//https://toph.co/p/play-the-lottery

#include <bits/stdc++.h>

using namespace std;

#define LL      long long
#define pii     pair<LL,LL>

const LL N= 1<<18;
const LL MOD=786433;

vector<LL>P[N];

LL rev[N],w[N|1],a[N],b[N],inv_n,g;

LL Pow(LL b,LL p){
    LL ret=1;
    while(p){
        if(p & 1) ret=(ret*b)%MOD;
        b=(b*b)%MOD;
        p>>=1;
    }
    return ret;
}

LL primitive_root(LL p){
    vector<LL>factor;
    LL phi = p-1,n=phi;

    for(LL i=2;i*i<=n;i++){
        if(n%i) continue;
        factor.emplace_back(i);
        while(n%i==0) n/=i;
    }

    if(n>1) factor.emplace_back(n);
    for(LL res=2;res<=p;res++){
        bool ok=true;
        for(LL i=0;i<factor.size() && ok;i++) ok &= Pow(res,phi/factor[i]) != 1;
        if(ok) return res;
    }
    return -1;
}

void prepare(LL n){
    LL sz=abs(31-__builtin_clz(n));
    LL r=Pow(g,(MOD-1)/n);
    inv_n=Pow(n,MOD-2);
    w[0]=w[n]=1;
    for(LL i=1;i<n;i++) w[i]= (w[i-1]*r)%MOD;
    for(LL i=1;i<n;i++) rev[i]=(rev[i>>1]>>1) | ((i & 1)<<(sz-1));
}

void NTT(LL *a,LL n,LL dir=0)
{
    for(LL i=1;i<n-1;i++) if(i<rev[i]) swap(a[i],a[rev[i]]);
    for(LL m=2;m<=n;m <<= 1) {
        for(LL i=0;i<n;i+=m){
            for(LL j=0;j< (m>>1);j++){
                LL &u=a[i+j],&v=a[i+j+(m>>1)];
                LL t=v*w[dir ? n-n/m*j:n/m*j]%MOD;
                v=u-t<0?u-t+MOD:u-t;
                u=u+t>=MOD?u+t-MOD:u+t;
            }
        }
    }
    if(dir) for(LL i=0;i<n;i++) a[i]=(inv_n*a[i])%MOD;
}

vector<LL> mul(vector<LL>p,vector<LL>q)
{
    LL n=p.size(),m=q.size();
    LL t=n+m-1,sz=1;
    while(sz<t) sz <<= 1;
    prepare(sz);

    for(LL i=0;i<n;i++) a[i]=p[i];
    for(LL i=0;i<m;i++) b[i]=q[i];

    for(LL i=n;i<sz;i++) a[i]=0;
    for(LL i=m;i<sz;i++) b[i]=0;

    NTT(a,sz);
    NTT(b,sz);
    for(LL i=0;i<sz;i++) a[i]=(a[i]*b[i])%MOD;
    NTT(a,sz,1);

    vector<LL> c(a,a+sz);
    while(c.size() && c.back()==0) c.pop_back();
    return c;
}

vector<LL> solve(LL l,LL r)
{
    if(l==r) return P[l];
    LL m=(l+r)/2;
    return mul(solve(l,m),solve(m+1,r));
}

int main()
{
    ios_base::sync_with_stdio(false);
    cin.tie(nullptr);
    LL m;
    cin >> m;
    for(LL i=1;i<=m;i++)
    {
        LL num;
        cin >> num;
        vector<pii>v;
        LL mx=0;
        while(num--)
        {
            LL typ,cnt;
            cin >> typ >> cnt;
            v.emplace_back(typ,cnt);
            mx=max(mx,typ);
        }
        P[i].resize(mx+1);
        for(pii p:v) P[i][p.first]=p.second;
    }
    g=primitive_root(MOD);
    vector<LL>c=solve(1,m);
    for(LL i=0;i<c.size();i++){
        if(c[i]){
            cout << i << ' ' << c[i] << '\n';
        }
    }
}
\end{lstlisting}
\subsection{Pollard Rho}
\begin{lstlisting}
LL mul(LL a,LL b,LL mod){
    return (__int128) a * b % mod;
    //LL ans = a * b - mod * (LL) (1.L / mod * a * b);
    //return ans + mod * (ans < 0) - mod * (ans >= (LL) mod);
}
LL bigmod(LL num,LL pow,LL mod){
    LL ans = 1;
    for( ;  pow > 0;  pow >>= 1, num = mul(num, num, mod))
        if(pow&1) ans = mul(ans,num,mod);
    return ans;
}
bool is_prime(LL n){
    if(n < 2 or n % 6 % 4 != 1) 
        return (n|1) == 3;
    LL a[] = {2, 325, 9375, 28178, 450775, 9780504, 1795265022};
    LL s = __builtin_ctzll(n-1), d = n >> s;
    for(LL x: a){
        LL p = bigmod(x % n, d, n), i = s;
        for( ; p != 1 and p != n-1 and x % n and i--; p = mul(p, p, n));
        if(p != n-1 and i != s)
            return false;
    }
    return true;
}
LL get_factor(LL n) {
    auto f = [&](LL x)  { return mul(x, x, n) + 1; };
    LL x = 0, y = 0, t = 0, prod = 2, i = 2, q;
    for(  ; t++ %40 or gcd(prod, n) == 1;   x = f(x), y = f(f(y)) ){
        (x == y) ? x = i++, y = f(x) : 0;
        prod = (q = mul(prod, max(x,y) - min(x,y), n)) ? q : prod;
    }
    return gcd(prod, n);
}
map <LL, int> factorize(LL n){
    map <LL, int> res;
    if(n < 2)   return res;
    LL small_primes[] = {2, 3, 5, 7, 11, 13, 17, 19, 23, 29, 31, 37, 41, 43, 47, 53, 59, 61, 67, 71, 73, 79, 83, 89, 97 };
    for (LL p: small_primes)
        for( ; n % p == 0; n /= p, res[p]++);

    auto _factor = [&](LL n, auto &_factor) {
        if(n == 1)   return;
        if(is_prime(n)) 
            res[n]++;
        else {
            LL x = get_factor(n);
            _factor(x, _factor);
            _factor(n / x, _factor);
        }
    };
    _factor(n, _factor);
    return res;
}
\end{lstlisting}
\subsection{Prime Counting Function}
\begin{lstlisting}
// initialize once by calling init()
#define MAXN 20000010       // initial sieve limit
#define MAX_PRIMES 2000010  // max size of the prime array for sieve
#define PHI_N 100000
#define PHI_K 100

int len = 0;  // total number of primes generated by sieve
int primes[MAX_PRIMES];
int pref[MAXN];        // pref[i] --> number of primes <= i
int dp[PHI_N][PHI_K];  // precal of yo(n,k)
bitset<MAXN> f;
void sieve(int n) {
    f[1] = true;
    for (int i = 4; i <= n; i += 2) f[i] = true;
    for (int i = 3; i * i <= n; i += 2) {
        if (!f[i]) {
            for (int j = i * i; j <= n; j += i << 1) f[j] = 1;
        }
    }
    for (int i = 1; i <= n; i++) {
        if (!f[i]) primes[len++] = i;
        pref[i] = len;
    }
}
void init() {
    sieve(MAXN - 1);
    // precalculation of phi upto size (PHI_N,PHI_K)
    for (int n = 0; n < PHI_N; n++) dp[n][0] = n;
    for (int k = 1; k < PHI_K; k++) {
        for (int n = 0; n < PHI_N; n++) {
            dp[n][k] = dp[n][k - 1] - dp[n / primes[k - 1]][k - 1];
        }
    }
}
// returns the number of integers less or equal n which are
// not divisible by any of the first k primes
// recurrence --> yo(n, k) = yo(n, k-1) - yo(n / p_k , k-1)
// for sum of primes yo(n, k) = yo(n, k-1) - p_k * yo(n / p_k , k-1)
long long yo(long long n, int k) {
    if (n < PHI_N && k < PHI_K) return dp[n][k];
    if (k == 1) return ((++n) >> 1);
    if (primes[k - 1] >= n) return 1;
    return yo(n, k - 1) - yo(n / primes[k - 1], k - 1);
}
// complexity: n^(2/3).(log n^(1/3))
long long Legendre(long long n) {
    if (n < MAXN) return pref[n];
    int lim = sqrt(n) + 1;
    int k = upper_bound(primes, primes + len, lim) - primes;
    return yo(n, k) + (k - 1);
}
// runs under 0.2s for n = 1e12
long long Lehmer(long long n) {
    if (n < MAXN) return pref[n];
    long long w, res = 0;
    int b = sqrt(n), c = Lehmer(cbrt(n)), a = Lehmer(sqrt(b));
    b = Lehmer(b);
    res = yo(n, a) + ((1LL * (b + a - 2) * (b - a + 1)) >> 1);
    for (int i = a; i < b; i++) {
        w = n / primes[i];
        int lim = Lehmer(sqrt(w));
        res -= Lehmer(w);
        if (i <= c) {
            for (int j = i; j < lim; j++) {
                res += j;
                res -= Lehmer(w / primes[j]);
            }
        }
    }
    return res;
}
\end{lstlisting}
\subsection{Shanks' Baby Step, Giant Step}
\begin{lstlisting}
// Finds a^x = b (mod p)

LL bigmod(LL b, LL p, LL m) {}

LL babyStepGiantStep(LL a, LL b, LL p)
{
    LL i, j, c, sq = sqrt(p);
    map<LL, LL> babyTable;

    for(j = 0, c = 1; j <= sq; j++, c = (c*a)%p)
        babyTable[c] = j;

    LL giant = bigmod(a, sq*(p-2), p);

    for(i = 0, c = 1; i <= sq; i++, c = (c*giant)%p)
    {
        if(babyTable.find((c*b)%p) != babyTable.end())
            return i*sq+babyTable[(c*b)%p];
    }

    return -1;
}
\end{lstlisting}
\subsection{Stirling Numbers}
\begin{lstlisting}
//stirling number 2nd kind variation(number of ways to place n marbles in k boxes so that each box has at least x marbles)
ll solve(int marble, int box) {
  if (marble < 1ll * box * x) return 0;
  if (box == 1 && marble >= x) return 1;
  if (vis[marble][box] == cs) return dp[marble][box];
  vis[marble][box] = cs;
  ll a = ( 1ll * box * solve(marble - 1, box) ) % MOD;
  ll b = ( 1ll * box * ncr(marble - 1, x - 1) ) % MOD;
  b = (b * solve(marble - x, box - 1)) % MOD;
  ll ret = (a + b) % MOD;
  return dp[marble][box] = ret;
}
//number of ways to place n marbles in k boxes so that no box is empty
ll stir(ll n, ll k) {
  ll ret = 0;
  for (int i = 0; i <= k; i++) {
    ll v = ncr(k, i) * bigmod(i, n) % MOD;
    if ( (k - i) % 2 == 0 ) ret = (ret + v) % MOD;
    else ret = (ret - v + MOD) % MOD;
  }
  return ret;
}
\end{lstlisting}
\subsection{Subset Convolution}
\begin{lstlisting}
inline int sgn(int mask) {
    return 1 - 2 * (__builtin_popcount(mask) & 1);
} // returns 1 if set cardinality is even, -1 otherwise

template <typename T, int b> struct Subset {
    static const int N = 1 << b;
    array <T, N> F;
    void Zeta() { // SOS
        for(int i = 0; i < b; i++)
            for(int mask = 0; mask < N; mask++)
                if(mask & 1 << i)
                    F[mask] += F[mask ^ 1 << i];
    }
    void OddEven() {
        for(int mask = 0; mask < N; mask++)
            F[mask] *= sgn(mask);
    }
    void MobiusOld() {
        OddEven();
        Zeta();
        OddEven();
    }
    void Mobius(){
        for(int i = 0; i < b; i++)
            for(int mask = 0; mask < N; mask++)
                if(mask & 1 << i)
                    F[mask] -= F[mask ^ 1 << i];
    }
    void operator *= (Subset &R) {
        auto &G = R.F;
        array < array <int, N>, b> Fh = {0}, Gh = {0}, H = {0};

        for(int mask = 0; mask < N; mask++) 
            Fh[__builtin_popcount(mask)][mask] = F[mask], Gh[__builtin_popcount(mask)][mask] = G[mask];

        for(int i = 0; i < b; i++)
            for(int j = 0; j < b; j++)
                for(int mask = 0; mask < N; mask++)
                    if((mask & (1 << j)) != 0)
                        Fh[i][mask] += Fh[i][mask ^ (1 << j)], Gh[i][mask] += Gh[i][mask ^ (1 << j)];
                    
        for(int mask = 0; mask < N; mask++)
            for(int i = 0; i < b; i++)
                for(int j = 0; j <= i; j++)
                    H[i][mask] += Fh[j][mask] * Gh[i - j][mask];

        for(int i = 0; i < b; i++) 
            for(int j = 0; j < b; j++) 
                for(int mask = 0; mask < N; mask++) 
                    if((mask & (1 << j)) != 0) 
                        H[i][mask] -= H[i][mask ^ (1 << j)];
                
        for(int mask = 0; mask < N; mask++)  
            F[mask] = H[__builtin_popcount(mask)][mask];
    }
    Subset operator * (Subset &R) {
        Subset ans = *this;
        return ans;
    }
};
\end{lstlisting}
\subsection{WalshHadamard}
\begin{lstlisting}
//CS Academy : Random Nim Generator

#include<bits/stdc++.h>
using namespace std;
typedef long long LL;
#define bitwiseXOR 1
//#define bitwiseAND 2
//#define bitwiseOR 3
const LL MOD = 30011;

LL BigMod(LL b,LL p)
{
    LL ret=1;
    while(p > 0){
        if(p % 2 == 1){
            ret=(ret*b)%MOD;
        }
        p = p/2;
        b=(b*b)%MOD;
    }
    return ret%MOD;
}

void FWHT(vector< LL >&p, bool inverse)
{
    LL n = p.size();
    assert((n&(n-1))==0);

    for (LL len = 1; 2*len <= n; len <<= 1) {
        for (LL i = 0; i < n; i += len+len) {
            for (LL j = 0; j < len; j++) {
                LL u = p[i+j];
                LL v = p[i+len+j];

                #ifdef bitwiseXOR
                p[i+j] = (u+v)%MOD;
                p[i+len+j] = (u-v+MOD)%MOD;
                #endif // bitwiseXOR

                #ifdef bitwiseAND
                if (!inverse) {
                    p[i+j] = v % MOD;
                    p[i+len+j] = (u+v) % MOD;
                } else {
                    p[i+j] = (-u+v) % MOD;
                    p[i+len+j] = u % MOD;
                }
                #endif // bitwiseAND

                #ifdef bitwiseOR
                if (!inverse) {
                    p[i+j] = u+v;
                    p[i+len+j] = u;
                } else {
                    p[i+j] = v;
                    p[i+len+j] = u-v;
                }
                #endif // bitwiseOR
            }
        }
    }

    #ifdef bitwiseXOR
    if (inverse) {
        LL val=BigMod(n,MOD-2); //Option 2: Exclude
        for (LL i = 0; i < n; i++) {
            //assert(p[i]%n==0); //Option 2: Include
            p[i] = (p[i]*val)%MOD; //Option 2: p[i]/=n;
        }
    }
    #endif // bitwiseXOR
}


int main()
{
    ios_base::sync_with_stdio(false);
    cin.tie(NULL);
    LL n ,k;
    cin >> n >> k;
    int len=1;
    while(len<=k) len <<= 1;
    vector<LL>a(len,0);
    for(int i=0;i<=k;i++) a[i]=1;
    FWHT(a,false);
    for(int i=0;i<len;i++) a[i]=BigMod(a[i],n);
    FWHT(a,true);
    LL ans=0;
    for(int i=1;i<a.size();i++) ans=(ans+a[i])%MOD;
    cout << ans%MOD;
}
\end{lstlisting}
\subsection{Xor Basis}
\begin{lstlisting}
struct XorBasis {
    static const int sz = 64;
    array <ULL, sz> base = {0}, back;
    array <int, sz> pos;
    void insert(ULL x, int p) {
        ULL cur = 0;
        for(int i = sz - 1; ~i; i--) if (x >> i & 1) {
            if(!base[i]) {
                base[i] = x, back[i] = cur, pos[i] = p;
                break;
            } else x ^= base[i], cur |= 1ULL << i;
        }
    }
    pair <ULL, vector <int>> construct(ULL mask) {
        ULL ok = 0, x = mask;
        for(int i = sz - 1; ~i; i--)  
            if(mask >> i & 1 and base[i])
                mask ^= base[i], ok |= 1ULL << i;
        vector <int> ans;
        for(int i = 0; i < sz; i++) if(ok >> i & 1) {
            ans.push_back(pos[i]);
            ok ^= back[i];
        }
        return {x ^ mask, ans};
    } 
};\end{lstlisting}
\section{String}
\subsection{Aho Corasick}
\begin{lstlisting}

const int sg = 26, N = 1e3 + 9;
struct aho_corasick {
    struct node{
        node *link, *out, *par;
        bool leaf;
        LL val;
        int cnt, last, len;
        char p_ch;
        array <node*, sg> to;
        node(node* par = NULL, char p_ch = '$', int len = 0): 
        par(par), p_ch(p_ch), len(len)  {
            val = leaf = cnt = last = 0;
            link = out = NULL;
        }
    };
    vector <node> trie;
    node *root;
    aho_corasick(){
        trie.reserve(N), trie.emplace_back();
        root = &trie[0];
        root-> link = root -> out = root;
    }
    inline int f(char c){
        return c - 'a';
    }
    inline node* add_node(node* par = NULL, char p_ch = '$', int len = 0){
        trie.emplace_back(par, p_ch, len);
        return &trie.back();
    }
    void add_str(const string& s, LL val = 1){
        node* now = root;
        for(char c: s){
            int i = f(c);
            if(!now->to[i])
                now->to[i] = add_node(now, c, now->len + 1);
            now = now -> to[i];
        }
        now -> leaf = true, now -> val++;

    }
    void push_links(){
        queue <node*> q;
        for(q.push(root); q.empty(); q.pop()){
            node *cur = q.front(), *link = cur -> link;
            cur -> out = link -> leaf ? link : link-> out;
            int idx = 0;
            for(auto &next: cur -> to) {
                if(next != NULL){
                    next -> link = cur != root ? link -> to[idx++] : root;
                    q.push(next);
                } 
                else next = link -> to[idx++]; 
            }
        }
        cur -> val += link -> val;
    }
};
\end{lstlisting}
\subsection{Double hash}
\begin{lstlisting}
ostream& operator << (ostream& os, PLL hash) {
  return os << "(" << hash.ff << ", " << hash.ss << ")";
}

PLL operator + (PLL a, LL x)     {return PLL(a.ff + x, a.ss + x);}
PLL operator - (PLL a, LL x)     {return PLL(a.ff - x, a.ss - x);}
PLL operator * (PLL a, LL x)     {return PLL(a.ff * x, a.ss * x);}
PLL operator + (PLL a, PLL x)    {return PLL(a.ff + x.ff, a.ss + x.ss);}
PLL operator - (PLL a, PLL x)    {return PLL(a.ff - x.ff, a.ss - x.ss);}
PLL operator * (PLL a, PLL x)    {return PLL(a.ff * x.ff, a.ss * x.ss);}
PLL operator % (PLL a, PLL m)    {return PLL(a.ff % m.ff, a.ss % m.ss);}

PLL base(1949313259, 1997293877);
PLL mod(2091573227, 2117566807);

PLL power (PLL a, LL p) {
  if (!p) return PLL(1, 1);
  PLL ans = power(a, p / 2);
  ans = (ans * ans) % mod;
  if (p % 2) ans = (ans * a) % mod;
  return ans;
}

PLL inverse(PLL a) {
  return power(a, (mod.ff - 1) * (mod.ss - 1) - 1);
}
PLL inv_base = inverse(base);

PLL val;
vector<PLL> P;

void hash_init(int n) {
  P.resize(n + 1);
  P[0] = PLL(1, 1);
  for (int i = 1; i <= n; i++) P[i] = (P[i - 1] * base) % mod;
}

///appends c to string
PLL append(PLL cur, char c) {
  return (cur * base + c) % mod;
}

///prepends c to string with size k
PLL prepend(PLL cur, int k, char c) {
  return (P[k] * c + cur) % mod;
}

///replaces the i-th (0-indexed) character from right from a to b;
PLL replace(PLL cur, int i, char a, char b) {
  cur = (cur + P[i] * (b - a)) % mod;
  return (cur + mod) % mod;
}

///Erases c from the back of the string
PLL pop_back(PLL hash, char c) {
  return (((hash - c) * inv_base) % mod + mod) % mod;
}

///Erases c from front of the string with size len
PLL pop_front(PLL hash, int len, char c) {
  return ((hash - P[len - 1] * c) % mod + mod) % mod;
}
///concatenates two strings where length of the right is k
PLL concat(PLL left, PLL right, int k) {
  return (left * P[k] + right) % mod;
}
///Calculates hash of string with size len repeated cnt times
///This is O(log n). For O(1), pre-calculate inverses
PLL repeat(PLL hash, int len, LL cnt) {
  PLL mul = (P[len * cnt] - 1) * inverse(P[len] - 1);
  mul = (mul % mod + mod) % mod;
  PLL ret = (hash * mul) % mod;

  if (P[len].ff == 1) ret.ff = hash.ff * cnt;
  if (P[len].ss == 1) ret.ss = hash.ss * cnt;
  return ret;
}
LL get(PLL hash) {
  return ( (hash.ff << 32) ^ hash.ss );
}
struct hashlist {
  int len;
  vector<PLL> H, R;

  hashlist() {}
  hashlist(string &s) {
    len = (int)s.size();
    hash_init(len);
    H.resize(len + 1, PLL(0, 0)), R.resize(len + 2, PLL(0, 0));
    for (int i = 1; i <= len; i++) H[i] = append(H[i - 1], s[i - 1]);
    for (int i = len; i >= 1; i--) R[i] = append(R[i + 1], s[i - 1]);
  }
  
  /// 1-indexed
  inline PLL range_hash(int l, int r) {
    int len = r - l + 1;
    return ((H[r] - H[l - 1] * P[len]) % mod + mod) % mod;
  }

  inline PLL reverse_hash(int l, int r) {
    int len = r - l + 1;
    return ((R[l] - R[r + 1] * P[len]) % mod + mod) % mod;
  }

  inline PLL concat_range_hash(int l1, int r1, int l2, int r2) {
    int len_2 = r2 - l2 + 1;
    return concat(range_hash(l1, r1), range_hash(l2, r2), len_2);
  }

  inline PLL concat_reverse_hash(int l1, int r1, int l2, int r2) {
    int len_1 = r1 - l1 + 1;
    return concat(reverse_hash(l2, r2), reverse_hash(l1, r1), len_1);
  }
};\end{lstlisting}
\subsection{Manacher's}
\begin{lstlisting}
#include <bits/stdc++.h>
using namespace std;

int main() {
    ios::sync_with_stdio(0);
    cin.tie(0);

    string s;
    cin >> s;

    int n = s.size();
    vector<int> d1(n);
    // d[i] = number of palindromes taking s[i] as center
    for (int i = 0, l = 0, r = -1; i < n; i++) {
        int k = (i > r) ? 1 : min(d1[l + r - i], r - i + 1);
        while (0 <= i - k && i + k < n && s[i - k] == s[i + k]) k++;
        d1[i] = k--;  if (i + k > r) l = i - k, r = i + k;
    }

    vector<int> d2(n);
    // d[i] = number of palindromes taking s[i-1] and s[i] as center
    for (int i = 0, l = 0, r = -1; i < n; i++) {
        int k = (i > r) ? 0 : min(d2[l + r - i + 1], r - i + 1);
        while (0 <= i - k - 1 && i + k < n && s[i - k - 1] == s[i + k]) k++;
        d2[i] = k--;   if (i + k > r) l = i - k - 1, r = i + k;
    }
}
\end{lstlisting}
\subsection{Palindromic Tree}
\begin{lstlisting}
struct state {
  int len, link;
  map<char, int> next;
};
state st[MAX];
int id, last;
string s;
ll ans[MAX];
void init() {
  for (int i = 0; i <= id; i++) {
    st[i].len = 0; st[i].link = 0;
    st[i].next.clear(); ans[i] = 0;
  }
  st[1].len = -1; st[1].link = 1;
  st[2].len = 0; st[2].link = 1;
  id = 2; last = 2;
}
void extend(int pos) {
  while (s[pos - st[last].len - 1] != s[pos]) last = st[last].link;
  int newlink = st[last].link;
  char c = s[pos];
  while (s[pos - st[newlink].len - 1] != s[pos]) newlink = st[newlink].link;
  if (!st[last].next.count(c)) {
    st[last].next[c] = ++id;
    st[id].len = st[last].len + 2;
    st[id].link = (st[id].len == 1 ? 2 : st[newlink].next[c]);
    ans[id] += ans[st[id].link];
    if (st[id].len > 2) {
      int l = st[id].len / 2 + (st[id].len % 2 ? 1 : 0);
      if (h.range_hash(pos - st[id].len + 1, pos - st[id].len + l) == h.reverse_hash(pos - st[id].len + 1, pos - st[id].len + l)) ans[id]++;
    }
  }
  last = st[last].next[c];
}
\end{lstlisting}
\subsection{String Match FFT}
\begin{lstlisting}
//find occurrences of t in s where '?'s are automatically matched with any character
//res[i + m - 1] = sum_j=0 to m - 1_{s[i + j] * t[j] * (s[i + j] - t[j])
vector<int> string_matching(string &s, string &t) {
  int n = s.size(), m = t.size();
  vector<int> s1(n), s2(n), s3(n);
  for(int i = 0; i < n; i++) s1[i] = s[i] == '?' ? 0 : s[i] - 'a' + 1; //assign any non zero number for non '?'s
  for(int i = 0; i < n; i++) s2[i] = s1[i] * s1[i];
  for(int i = 0; i < n; i++) s3[i] = s1[i] * s2[i];
  vector<int> t1(m), t2(m), t3(m);
  for(int i = 0; i < m; i++) t1[i] = t[i] == '?' ? 0 : t[i] - 'a' + 1;
  for(int i = 0; i < m; i++) t2[i] = t1[i] * t1[i];
  for(int i = 0; i < m; i++) t3[i] = t1[i] * t2[i];
  reverse(t1.begin(), t1.end());
  reverse(t2.begin(), t2.end());
  reverse(t3.begin(), t3.end());
  vector<int> s1t3 = multiply(s1, t3);
  vector<int> s2t2 = multiply(s2, t2);
  vector<int> s3t1 = multiply(s3, t1);
  vector<int> res(n);
  for(int i = 0; i < n; i++) res[i] = s1t3[i] - s2t2[i] * 2 + s3t1[i];
  vector<int> oc;
  for(int i = m - 1; i < n; i++) if(res[i] == 0) oc.push_back(i - m + 1);
  return oc;
}
\end{lstlisting}
\subsection{Suffix Array}
\begin{lstlisting}

/**
 Suffix Array implementation with count sort.
 Source: E-MAXX
 Running time:
    Suffix Array Construction: O(NlogN)
    LCP Array Construction: O(NlogN)
    Suffix LCP: O(logN)
**/

#include <bits/stdc++.h>
using namespace std;

typedef pair<int, int> PII;
typedef vector<int> VI;

/// Equivalence Class INFO
vector<VI> c;
VI sort_cyclic_shifts(const string &s) {
    int n = s.size();
    const int alphabet = 256;
    VI p(n), cnt(alphabet, 0);

    c.clear();
    c.emplace_back();
    c[0].resize(n);

    for (int i = 0; i < n; i++) cnt[s[i]]++;
    for (int i = 1; i < alphabet; i++) cnt[i] += cnt[i - 1];
    for (int i = 0; i < n; i++) p[--cnt[s[i]]] = i;

    c[0][p[0]] = 0;
    int classes = 1;

    for (int i = 1; i < n; i++) {
        if (s[p[i]] != s[p[i - 1]]) classes++;
        c[0][p[i]] = classes - 1;
    }

    VI pn(n), cn(n);
    cnt.resize(n);

    for (int h = 0; (1 << h) < n; h++) {
        for (int i = 0; i < n; i++) {
            pn[i] = p[i] - (1 << h);
            if (pn[i] < 0) pn[i] += n;
        }
        fill(cnt.begin(), cnt.end(), 0);

        /// radix sort
        for (int i = 0; i < n; i++) cnt[c[h][pn[i]]]++;
        for (int i = 1; i < classes; i++) cnt[i] += cnt[i - 1];
        for (int i = n - 1; i >= 0; i--) p[--cnt[c[h][pn[i]]]] = pn[i];

        cn[p[0]] = 0;
        classes = 1;

        for (int i = 1; i < n; i++) {
            PII cur = {c[h][p[i]], c[h][(p[i] + (1 << h)) % n]};
            PII prev = {c[h][p[i - 1]], c[h][(p[i - 1] + (1 << h)) % n]};
            if (cur != prev) ++classes;
            cn[p[i]] = classes - 1;
        }
        c.push_back(cn);
    }
    return p;
}

VI suffix_array_construction(string s) {
    s += "!";
    VI sorted_shifts = sort_cyclic_shifts(s);
    sorted_shifts.erase(sorted_shifts.begin());
    return sorted_shifts;
}

/// LCP between the ith and jth (i != j) suffix of the STRING
int suffixLCP(int i, int j) {
    assert(i != j);
    int log_n = c.size() - 1;

    int ans = 0;
    for (int k = log_n; k >= 0; k--) {
        if (c[k][i] == c[k][j]) {
            ans += 1 << k;
            i += 1 << k;
            j += 1 << k;
        }
    }
    return ans;
}

VI lcp_construction(const string &s, const VI &sa) {
    int n = s.size();
    VI rank(n, 0);
    VI lcp(n - 1, 0);

    for (int i = 0; i < n; i++) rank[sa[i]] = i;

    for (int i = 0, k = 0; i < n; i++) {
        if (rank[i] == n - 1) {
            k = 0;
            continue;
        }

        int j = sa[rank[i] + 1];
        while (i + k < n && j + k < n && s[i + k] == s[j + k]) k++;
        lcp[rank[i]] = k;
        if (k) k--;
    }
    return lcp;
}

const int MX = 1e6 + 7, K = 20;
int lg[MX];

void pre() {
    lg[1] = 0;
    for (int i = 2; i < MX; i++) lg[i] = lg[i / 2] + 1;
}

struct RMQ {
    int N;
    VI v[K];
    RMQ(const VI &a) {
        N = a.size();
        v[0] = a;

        for (int k = 0; (1 << (k + 1)) <= N; k++) {
            v[k + 1].resize(N);
            for (int i = 0; i - 1 + (1 << (k + 1)) < N; i++) {
                v[k + 1][i] = min(v[k][i], v[k][i + (1 << k)]);
            }
        }
    }

    int findMin(int i, int j) {
        int k = lg[j - i + 1];
        return min(v[k][i], v[k][j + 1 - (1 << k)]);
    }
};
\end{lstlisting}
\subsection{Suffix Automata}
\begin{lstlisting}

/**
    Linear Time Suffix Automata contruction.
    Build Complexity: O(n * alphabet)
    To achieve better build complexity and linear space,
    use map for transitions.
**/

#include<bits/stdc++.h>
using namespace std;

const int MAXN = 1e5+7, ALPHA = 26;
int len[2*MAXN], link[2*MAXN], nxt[2*MAXN][ALPHA];
int sz;
int last;

void sa_init() {
    memset(nxt, -1, sizeof nxt);

    len[0] = 0;
    link[0] = -1;
    sz = 1;
    last = 0;
}

void add(char ch) {
    int c = ch-'a';

    int cur = sz++;                             //create new node
    len[cur] = len[last]+1;

    int u = last;
    while (u != -1 && nxt[u][c] == -1) {
        nxt[u][c] = cur;
        u = link[u];
    }

    if (u == -1) {
        link[cur] = 0;
    }
    else {
        int v = nxt[u][c];
        if (len[v] == len[u]+1) {
            link[cur] = v;
        }
        else {
            int clone = sz++;                   //create node by cloning
            len[clone] = 1 + len[u];
            link[clone] = link[v];

            for (int i=0; i<ALPHA; i++)
                nxt[clone][i] = nxt[v][i];

            while (u != -1 && nxt[u][c] == v) {
                nxt[u][c] = clone;
                u = link[u];
            }

            link[v] = link[cur] = clone;
        }
    }
    last = cur;
}

vector<int> edge[2*MAXN];
///Optional, Call after adding all characters
void makeEdge() {
    for (int i=0; i<sz; i++) {
        edge[i].clear();
        for (int j=0; j<ALPHA; j++)
            if (nxt[i][j]!=-1)
                edge[i].push_back(j);
    }
}

// The following code solves SPOJ SUBLEX
// Given a string S, you have to answer some queries:
// If all distinct substrings of string S were sorted
// lexicographically, which one will be the K-th smallest?

long long dp[2*MAXN];
bool vis[2*MAXN];

void dfs(int u) {
    if (vis[u]) return;
    vis[u] = 1;
    dp[u] = 1;
    for (int i: edge[u]) {
        if (nxt[u][i] == -1)    continue;
        dfs(nxt[u][i]);
        dp[u] += dp[nxt[u][i]];
    }
}

void go(int u, long long rem, string &s) {
    if (rem == 1)   return;
    long long sum = 1;
    for (int i: edge[u]) {
        if (nxt[u][i] == -1)    continue;
        if (sum + dp[nxt[u][i]] < rem) {
            sum += dp[nxt[u][i]];
        }
        else {
            s += ('a' + i);
            go(nxt[u][i], rem-sum, s);
            return;
        }
    }
}

int main() {
    ios::sync_with_stdio(0);
    cin.tie(0);

    string s;
    cin>>s;

    sa_init();
    for (char c: s) add(c);
    makeEdge();


    dfs(0);
    int q;
    cin>>q;

    while (q--) {
        long long x;
        cin>>x;
        x++;
        string s;
        go(0, x, s);
        cout<<s<<"\n";
    }
}
\end{lstlisting}
\subsection{Z Algo}
\begin{lstlisting}
vector<int> calcz(string s) {
    int n = s.size();
    vector<int> z(n);
    int l, r; l = r = 0;
    for (int i = 1; i < n; i++) {
        if (i > r) {
            l = r = i;
            while (r < n && s[r] == s[r - l]) r++;
            z[i] = r - l; r--;
        } else {
            int k = i - l;
            if (z[k] < r - i + 1) z[i] = z[k];
            else {
                l = i;
                while (r < n && s[r] == s[r - l]) r++;
                z[i] = r - l; r--;
            }
        }
    }
    return z;
}
\end{lstlisting}
\end{multicols*}
\begin{multicols*}{3}
\newpage
\section{Equations and Formulas}
\subsection{Catalan Numbers}
$\displaystyle C_n=\frac{1}{n+1}{2n \choose n}$
$\displaystyle C_0=1,C_1=1\text{ and }C_n=\sum \limits_{k=0}^{n-1}C_k C_{n-1-k}$ \\
The number of ways to completely parenthesize $n$+$\displaystyle 1$ factors. \\
The number of triangulations of a convex polygon with $n$+$\displaystyle 2$ sides (i.e. the number of partitions of polygon into disjoint triangles by using the diagonals). \\
The number of ways to connect the $\displaystyle 2n$ points on a circle to form $n$ disjoint i.e. non-intersecting chords. \\
The number of rooted full binary trees with $n$+$\displaystyle 1$ leaves (vertices are not numbered). A rooted binary tree is full if every vertex has either two children or no children. \\
Number of permutations of $\displaystyle {1, …, n}$ that avoid the pattern $\displaystyle 123$ (or any of the other patterns of length $3$); that is, the number of permutations with no three-term increasing sub-sequence. For $n = 3$, these permutations are $\displaystyle 132,\ 213,\ 231,\ 312$ and $321.$

\subsection{Stirling Numbers First Kind}
The Stirling numbers of the first kind count permutations according to their number of cycles (counting fixed points as cycles of length one). \\
$S(n,k)$ counts the number of permutations of $n$ elements with $\displaystyle \displaystyle k$ disjoint cycles. \\
$S(n,k)=(n-1) \cdot S(n-1,k)+S(n-1,k-1),$ \(where,\; S(0,0)=1,S(n,0)=S(0,n)=0\)
$\displaystyle \displaystyle\sum_{k=0}^{n}S(n,k) = n!$ \\
The unsigned Stirling numbers may also be defined algebraically, as the coefficient of the rising factorial:
\[\displaystyle x^{\bar{n}} = x(x+1)...(x+n-1) = \sum_{k=0}^{n}{ S(n, k) x^k}\]
Lets $[n, k]$ be the stirling number of the first kind, then

\[\displaystyle \bigl[\!\begin{smallmatrix} n \\ n\ -\ k \end{smallmatrix}\!\bigr] = \sum_{0 \leq i_1 < i_2 < i_k < n}{i_1i_2....i_k.}\]

\subsection{Stirling Numbers Second Kind}
Stirling number of the second kind is the number of ways to partition a set of n objects into k non-empty subsets. \\
$S(n,k)=k \cdot S(n-1,k)+S(n-1,k-1)$, \(where \; S(0,0)=1,S(n,0)=S(0,n)=0\)
$S(n,2)=2^{n-1}-1$ 
$S(n,k) \cdot k!$ = number of ways to color $n$ nodes using colors from $\displaystyle 1$ to $\displaystyle \displaystyle k$ such that each color is used at least once. \\
An $r$-associated Stirling number of the second kind is the number of ways to partition a set of $n$ objects into $\displaystyle \displaystyle k$ subsets, with each subset containing at least $r$ elements. It is denoted by $S_r( n , k )$ and obeys the recurrence relation. $\displaystyle \displaystyle S_r(n+1,k) = k S_r(n,k) + \binom{n}{r-1} S_r(n-r+1,k-1)$ \\ 
Denote the n objects to partition by the integers $\displaystyle 1, 2, …., n$. Define the reduced Stirling numbers of the second kind, denoted $S^d(n, k)$, to be the number of ways to partition the integers $\displaystyle 1, 2, …., n$ into k nonempty subsets such that all elements in each subset have pairwise distance at least d. That is, for any integers i and j in a given subset, it is required that $|i - j| \geq d$. It has been shown that these numbers satisfy, \(S^d(n, k) = S(n - d + 1, k - d + 1), n \geq k \geq d\)
\subsection{Other Combinatorial Identities}
$\displaystyle \displaystyle {n \choose k}=\frac{n}{k}{n-1 \choose k-1}$ \\
$\displaystyle \sum \limits_{i= 0}^k{n+i \choose i}= \sum \limits_{i= 0}^k{n+i \choose n} = {n+k+1 \choose k}$ \\
$\displaystyle \ n,r \in N, n > r, \sum \limits_{i=r}^n{i \choose r}={n+1 \choose r+1}$ \\
If $\displaystyle P(n)=\sum_{k=0}^{n}{n \choose k} \cdot Q(k)$, then,
\[Q(n)=\sum_{k=0}^{n}(-1)^{n-k}{n \choose k} \cdot P(k)\] \\
If $\displaystyle P(n)=\sum_{k=0}^{n}(-1)^{k}{n \choose k} \cdot Q(k)$ , then,
\[Q(n)=\sum_{k=0}^{n}(-1)^{k}{n \choose k} \cdot P(k)\]

\subsection{Different Math Formulas}
\textbf{Picks Theorem : } $ A = i + b / 2 - 1 $ \\ 
\textbf{Deragements : } $ d(i) = (i - 1) \times \left( d(i - 1) + d(i - 2) \right) $ \\ 
\begin{multline*}
\displaystyle \frac{n}{ab}-\Big\{\frac{b{\prime} n}{a}\Big\}-\Big\{\frac{a{\prime} n}{b}\Big\} + 1
\end{multline*}
\subsection{GCD and LCM}
if $m$ is any integer, then $\displaystyle \gcd(a + m {\cdot} b, b) = \gcd(a, b)$ \\
The gcd is a multiplicative function in the following sense: if $\displaystyle a_1$ and $\displaystyle a_2$ are relatively prime, then $\displaystyle \gcd(a_1 \cdot a_2, b) = \gcd(a_1, b) \cdot \gcd(a_2,b )$. \\
$\displaystyle \gcd(a, \operatorname{lcm}(b, c)) = \operatorname{lcm}(\gcd(a, b), \gcd(a, c))$. \\
$\displaystyle \operatorname{lcm}(a, \gcd(b, c)) = \gcd(\operatorname{lcm}(a, b), \operatorname{lcm}(a, c))$. \\
For non-negative integers $\displaystyle a$ and $b$, where $\displaystyle a$ and $b$ are not both zero, $\displaystyle \gcd({n^a} - 1, {n^b} - 1) = n^{\gcd(a,b)} - 1$ \\
$\displaystyle \gcd(a, b) = \displaystyle \sum_{k|a \, \text{and} \, k|b} {\phi(k)}$ \\
$\displaystyle \displaystyle \sum_{i=1}^n [\gcd(i, n) = k] = { \phi{\left(\frac{n}{k}\right)}}$ \\
$\displaystyle \displaystyle \sum_{k=1}^n \gcd(k, n) = \displaystyle \sum_{d|n} d \cdot {\phi{\left(\frac{n}{d}\right)}}$ \\
$\displaystyle \displaystyle \sum_{k=1}^n x^{\gcd(k,n)} = \displaystyle \sum_{d|n} x^d \cdot {\phi{\left(\frac{n}{d}\right)}}$ \\
$\displaystyle \displaystyle \sum_{k=1}^n \frac{1}{\gcd(k, n)} = \displaystyle \sum_{d|n} \frac{1}{d} \cdot {\phi{\left(\frac{n}{d}\right)}} = \frac{1}{n} \displaystyle \sum_{d|n} d \cdot \phi(d)$ \\
$\displaystyle \displaystyle \sum_{k=1}^n \frac{k}{\gcd(k, n)} = \frac{n}{2} \cdot \displaystyle \sum_{d|n} \frac{1}{d} \cdot {\phi{\left(\frac{n}{d}\right)}} = \frac{n}{2} \cdot \frac{1}{n} \cdot \displaystyle \sum_{d|n} d \cdot \phi(d)$ \\
$\displaystyle \displaystyle \sum_{k=1}^n \frac{n}{\gcd(k, n)} = 2 * \displaystyle \sum_{k=1}^n \frac{k}{\gcd(k, n)} - 1$, for $n > 1$ \\
$\displaystyle \displaystyle \sum_{i=1}^n \sum_{j=1}^n [\gcd(i, j) = 1] = \displaystyle \sum_{d=1}^n \mu(d) \lfloor {\frac{n}{d} \rfloor}^2$ \\
$\displaystyle \displaystyle \sum_{i=1}^n \displaystyle\sum_{j=1}^n \gcd(i, j) = \displaystyle \sum_{d=1}^n \phi(d) \lfloor {\frac{n}{d} \rfloor}^2$ \\
$\displaystyle \sum_{i=1}^n \sum_{j=1}^n i \cdot j[\gcd(i, j) = 1] = \sum_{i=1}^n \phi(i)i^2$ \\
$\displaystyle F(n) = \displaystyle \sum_{i=1}^n \displaystyle \sum_{j=1}^n \operatorname{lcm}(i, j) = \displaystyle \sum_{l=1}^n {\left(\frac{\left( 1 + \lfloor{\frac{n}{l} \rfloor} \right) \left( \lfloor{\frac{n}{l} \rfloor} \right)} {2} \right)}^2 \displaystyle \sum_{d|l} \mu(d)ld$ \\
\end{multicols*}

\end{document}
